\documentclass{article}
\usepackage{arxiv}

\usepackage[utf8]{inputenc} % allow utf-8 input
\usepackage[T1]{fontenc}    % use 8-bit T1 fonts
\usepackage{hyperref}       % hyperlinks
\usepackage{url}            % simple URL typesetting
\usepackage{booktabs}       % professional-quality tables
\usepackage{amsfonts}       % blackboard math symbols
\usepackage{amsmath}        % why is this not included by default?
\usepackage{nicefrac}       % compact symbols for 1/2, etc.
\usepackage{microtype}      % microtypography
\usepackage{xcolor}      % colored text
\usepackage{tikz}
\usepackage{cleveref}       % smart cross-referencing
\usepackage{lipsum}         % Can be removed after putting your text content
\usepackage{graphicx}
\usepackage{natbib}
\usepackage{doi}

\usetikzlibrary{shapes.geometric, arrows}
\tikzstyle{startstop} = [rectangle, rounded corners=2mm, minimum width=3cm, minimum height=1cm,text centered, draw=black, fill=white!30]
\tikzstyle{inner} = [rectangle, rounded corners=2mm, minimum width=3cm, minimum height=1cm,text centered, draw=black, fill=white!30]
\tikzstyle{dimensions} = [rectangle, rounded corners, minimum width=3cm, minimum height=1cm,text centered, draw=black, fill=white!30]
\tikzstyle{arrow} = [thick,->,>=stealth]

\title{Cortical Morphological Congruence as a Biomarker of Brain Development Assessed with Magnetic Resonance Imaging}

% Here you can change the date presented in the paper title
%\date{September 9, 1985}
% Or remove it
%\date{}


\newif\ifuniqueAffiliation
% Comment to use multiple affiliations variant of author block
% \uniqueAffiliationtrue

\ifuniqueAffiliation % Standard variant of author block
\author{%
	\href{https://orcid.org/0000-0002-9604-3157}{\includegraphics[scale=0.06]{orcid.pdf}\hspace{1mm}Jacob Levman}
	\thanks{Use footnote for providing further information about author (webpage, alternative address)---\emph{not} for acknowledging funding agencies.} \\
	Department of Computer Science\\
	St. Francis Xavier University\\
	Antigonish, Nova Scotia, Canada\\
	Athinoula A. Martinos Center for Biomedical Imaging, Charlestown, Massachusetts 02129\\
	\texttt{jlevman@stfx.ca} \\

	\And

	\href{https://orcid.org/0000-0003-4733-0624}{\includegraphics[scale=0.06]{orcid.pdf}\hspace{1mm}Derek Berger} \\
	Department of Computer Science\\
	St. Francis Xavier University\\
	Antigonish, Nova Scotia, Canada\\
	\texttt{dberger@stfx.ca} \\
	%% \AND
	%% Coauthor \\
	%% Affiliation \\
	%% Address \\
	%% \texttt{email} \\
}
\else
% Multiple affiliations variant of author block
\usepackage{authblk}
\renewcommand\Authfont{\bfseries}
\setlength{\affilsep}{0em}
% box is needed for correct spacing with authblk
\newbox{\orcid}\sbox{\orcid}{\includegraphics[scale=0.06]{orcid.pdf}}
\author[1,2]{%
	\href{https://orcid.org/0000-0002-9604-3157}{\usebox{\orcid}\hspace{1mm}Jacob Levman\thanks{\texttt{jlevman@stfx.ca}}}%
}
\author[1]{%
	\href{https://orcid.org/0000-0003-4733-0624}{\usebox{\orcid}\hspace{1mm}Derek Berger\thanks{\texttt{dberger@stfx.ca}}}%
}
\affil[1]{Athinoula A. Martinos Center for Biomedical Imaging, Charlestown, Massachusetts 02129}
\affil[2]{Department of Computer Science, St. Francis Xavier University,Antigonish, Nova Scotia}
\affil[3]{Nova Scotia Health Authority, Halifax, NS, Canada}
\affil[4]{Harvard Medical School, Department of Radiology, Boston, MA, USA.}
\fi

% Uncomment to override  the `A preprint' in the header
%\renewcommand{\headeright}{Technical Report}
%\renewcommand{\undertitle}{Technical Report}
\renewcommand{\shorttitle}{\textit{arXiv} Cortical Congruence}

%%% Add PDF metadata to help others organize their library
%%% Once the PDF is generated, you can check the metadata with
%%% $ pdfinfo template.pdf
\hypersetup{pdftitle={Cortical Congruence},
pdfsubject={q-bio.NC, q-bio.QM},
pdfauthor={Jacob Levman, Derek Berger},
pdfkeywords={morphological, congruence, cortex, neurodevelopment, magnetic resonance imaging, healthy},
}

\begin{document}
\maketitle

\begin{abstract}
A set of novel biomarkers are proposed for the extraction of regional
cortical morphological congruence (CMC) measurements from neurological
magnetic resonance imaging examinations. These proposed novel biomarkers
assess a cortical region's deviation from expectation for a perfectly regular
structure. The proposed CMC biomarkers have been applied to a large sample
(n=1113) of neurological MRI examinations of neurotypical subjects from the
Human Connectome Project. Results demonstrate consistent patterns of regional
brain differences in CMC across the cortex, implying differential development
regionally in the healthy brain characterizable by CMC\@. Results demonstrate
relatively small standard deviations of CMC values across a large population,
implying potential for these biomarkers as a reliable and easily reproducible
method to characterize brain development. \textcolor{orange}{Perhaps a
sentence here on similarities and differences between males and females as
part of this baseline analysis. Perhaps Sentence on some other stuff, like
our ability to predict IQ or something from CMC measurement}. Future
work will investigate CMC's potential to further characterize healthy brain
development, as well as to characterize a variety of different pathological
conditions.
\end{abstract}


% keywords can be removed
\keywords{morphological \and congruence \and cortex \and neurodevelopment \and magnetic resonance imaging \and healthy}


\section{Introduction}
Characterization of human cortical development in vivo requires medical
imaging technology that provides tissue contrast between gray and white
matter. Magnetic resonance imaging (MRI) is sensitive to hydrogen proton
concentration, which is variable across tissues, thus MRI provides excellent
soft tissue contrast, including between the gray and white matter in the
brain \citep{duboisMRINeonatalBrain2021}. Automated methods for extracting
biomarkers of potential interest, such as a regional cortical tissue's volume
(mm3), surface area (mm2) or thickness (mm), have long been relied upon for
the study of the human brain
\citep{fischlFreeSurfer2012,levmanPediatricStructuralMRI2017,levmanStructuralMagneticResonance2019a,mccannStructuralMagneticResonance2021}.
However, the variability of these volume, surface area, and thickness
measurements across a given population is known to be quite large
\citep{fischlFreeSurfer2012,levmanPediatricStructuralMRI2017,levmanStructuralMagneticResonance2019a},
potentially contributing to known reproducibility challenges in modern
neuroscience studies
\citep{martinezReproducibilityBraincognitionRelationships2015,marekReproducibleBrainwideAssociation2022},
and may be part of the reason that these techniques are generally not yet
relied upon for clinical characterization.

Very broadly, congruence is analogous to agreement between two or more
objects, studies, shapes, individual measurements, etc. For instance, the
results of one study may be congruent with those already in the literature,
or two objects are deemed congruent if they have the same shape and size.
Congruence can be applied in many ways, and has been the subject of limited
and diverse studies focused on the human brain. Research has suggested that
the development of visual cortical properties is dependent on
visuo-proprioceptive congruence \citep{buisseretChapter22Development1993}.
More recently, a model has been specifically developed for congruence of
binocular vision (how information from the left and right eye are
incorporated) in the primary visual cortex
\citep{somaratnaModelDevelopmentBinocular2022}. Congruence has also been
assessed between interoceptive predictions and hippocampal-related memory
\citep{edwards-duricCongruenceInteroceptivePredictions2020}. Congruence
between the development of the circulatory and nervous systems, or
neurovascular congruence, has been the subject of a study focused on cortical
development \citep{stubbsNeurovascularCongruenceCerebral2009}. Additionally,
it has been reported that congruence based contextual plausibility modulates
cortical activity during vibrotactile perception
\citep{kangCongruencebasedContextualPlausibility2022}. Neuronal congruency
has also been assessed in the macaque prefrontal cortex
\citep{yaoNeuronalCongruencyEffects2022}. This manuscript presents a novel
set of biomarkers for characterization of regional cortical morphological
congruence (CMC), which can be referred to more simply as cortical congruence
(CC). The proposed methods assess the degree of congruence between multiple
cortical measurements, thus providing novel biomarkers which we hypothesize
may help characterize neurodevelopment.


\section{Methods}
\label{sec:methods}
\subsection{Subject Populations and Imaging}
\label{sec:population}

Data were provided in part by the Human Connectome Project, WU-Minn
Consortium (Principal Investigators: David Van Essen and Kamil Ugurbil;
1U54MH091657) funded by the 16 NIH Institutes and Centers that support the
NIH Blueprint for Neuroscience Research; and by the McDonnell Center for
Systems Neuroscience at Washington University. This cohort included 1,113
healthy subjects imaged with MRI\@. Detailed information on the magnetic
resonance imaging (MRI) scanners and protocols used in the Human Connectome
Project dataset are available in the literature
\citep{elamHumanConnectomeProject2021}.

\subsection{Postprocessing}
\label{sec:postprocessing}

The Human Connectome Project's WU-Minn HCP cohort (n=1,113 with MRI
examinations) was processed by FreeSurfer \citep{fischlFreeSurfer2012} and
the results were made publicly available through the Human Connectome
Project's \href{%
https://www.humanconnectome.org/study/hcp-young-adult/document/1200-subjects-data-release}{website}.


\subsubsection{Congruence Features}
\label{sec:cmc-features}

For each cortical region of interest (ROI) \(r\) (e.g.
\(r=\text{precuneus}\)) supported by the publicly available FreeSurfer
results (see Appendix Tbl lateral\_fs for the base FreeSurfer features),
Cortical Morphological Congruence (CMC) measurements were computed in each
patient.

Define \(V(r)\) to be the volume of region \(r\), and \(A(r)\) and \(d(r)\)
to be the surface area and average thickness, respectively, of the ROI\@.
Denote the left and right ROIs for \(r\) as \(r_l\) and \(r_r\) respectively
(e.g\@. if \(r=\text{precuneus}\), then \(r_l\) and \(r_r\) are the left and
right precuneus). Then define the *lateral* CMC metrics to be:

\begin{equation}
\text{CMC}(r) = \text{CMC}_{\text{Lateral}}(r) = \frac{V(r)}{A(r) \cdot d(r)}
\end{equation}


Note that \(A(r) \cdot d(r)\) is the product of an area and thickness: we thus
denote the *pseudo-volume*:

\begin{equation}
\hat{V}(r) = \text{CMC}_{pV}(r) = V(r) \cdot d(r)
\end{equation}

Then:

\begin{align*}
\text{CMC}_l &= \text{CMC}_{\text{Lateral}}(r_l) \\
\text{CMC}_r &= \text{CMC}_{\text{Lateral}}(r_r) \\
\end{align*}
% {#eq:cmc_laterals}

Also define "bilateral" CMC as:

\begin{equation}
\text{CMC}_{lr}
=\frac{V(r_l) + V(r_r)}{\hat{V}(r_l) + \hat{V}(r_r)}
=\frac{\text{ROI total volume}}{\text{ROI total pseudo-volume}}
% $$ {#eq:cmc_bilateral}
\end{equation}

Define the *asymmetric* CMC measures:

\[ \text{CMC}_{l - r} = \text{CMC}_{l} - \text{CMC}_{r} \]
% {#eq:asym_signed_diff}

and

\[ \text{CMC}_{|l - r|} = \lvert\text{CMC}_{l - r} \rvert \]
% {#eq:asym_unsigned_diff}

which are the *signed* and *unsigned CMC differences*, respectively. Also define

\[ \text{CMC}_{l / r} = \text{CMC}_{l} / \text{CMC}_{r} \]
% {#eq:asym_ratio}

to be the asymmetric CMC *ratio* measure.

Then, for each ROI \(r\), this defines seven \(\text{CMC}\) measures:
\(\text{CMC}_{pV}\), \(\text{CMC}_l\), \(\text{CMC}_r\), \(\text{CMC}_{lr}\),
\(\text{CMC}_{l-r}\), \(\text{CMC}_{\lvert l-r \rvert}\), and \(\text{CMC}_{l / r}\).

The CMC features defined in eq cmc\_laterals and eq cmc\_bilateral are unitless
ratios assessing a cortical region's deviation from uniform thickness, where
the region pseudo-volume would be expected to be equal to the surface area
times the average cortical thickness. In the case of equality, this produces
a lateral CMC biomarker value of 1. Deviations from 1 in either direction
have major implications for structural cortical presentation, imply
differential neurological development has occurred, and is addressed in
detail in the Discussion.

Eq asym\_signed\_diff defines a biomarker that preserves directionality of
asymmetry, with left/right dominant regions taking on positive/negative
values of \(\text{CMC}_{l - r}\), respectively. Likewise, eq:asym\_ratio
defines a quantity which is greater/less than \(1\) when left/right lateral
\(\text{CMC}\) is larger.

\subsection{Statistical and Predictive Analyses}

For each of the 34 cortical regions \(r\) available from the FreeSurfer
analysis (see Table~\ref{tab:lateral-cmc}), each of the seven CMC features was
computed for each subject. Then, we run run various statistical and
predictive analyses based on these extracted features.

\subsubsection{Descriptive Statistics}

\textcolor{orange}{For all the stats below, we could also alternately
estimate CIs on the Cohen's d values via a percentile bootstrap, if you don't
like using Mann-Whitney and Wilcoxon signed-rank tests.}


\paragraph{Sex}: For each CMC feature, we compute descriptive measures of
group separation by sex using Cohen's d and the Mann-Whitney U test. We also
consider each of the seven classes of CMC feature, and compare whether these
feature classes differ in variability. This is done by computing the average
CMC metric in each CMC class for each group (e.g.\ sex, laterality, age), and
then comparing these two group averages with a measure of scale (standard
deviation; SD, or interquartile range; IQR).

\paragraph{Laterality}: For each ROI, it is reasonable to ask whether
\(\text{CMC}_l\) and \(\text{CMC}_r\) differ. We also compare relevant CMC
features in this manner.

\paragraph{Age and Sex}: For each CMC feature in each CMC feature class, we
compute the Spearman correlations with subject age class (HCP phenotypic data
includes only broad age bins, and thus requires a correlation appropriate for
ordinal variables).

\subsubsection{Predictive Analyses}

The HCP data provides a wealth of
\href{https://www.humanconnectome.org/study/hcp-young-adult/document/wu-minn-hcp-consortium-open-access-data-use-terms}{openly-accessible
behavioural data} on participating subjects. Full details of the hundreds of
features are
\href{https://wiki.humanconnectome.org/docs/HCP-YA%20Data%20Dictionary-%20Updated%20for%20the%201200%20Subject%20Release.html}{available
online}.

\paragraph{Synthetic Behavioural Targets} \label{sec:cleaning}

To investigate the predictive potential of CMC features, a number of \emph{ad hoc}
regression targets were constructed from the HCP behavioural data. First,
behavioural data were cleaned of irrelevant, missing, constant, or redundant
features (see sec:cleaning). (\texttt{src/munging/hcp.py::load\_hcp\_phenotypic}).

\begin{figure}
\centering
\footnotesize

\begin{tikzpicture}[node distance=2cm]
\node (start) {Behavioural data initial shape};
\node (s1) [below of=start, yshift=1cm] {(1206, 582)};
\node (c1) [below of=s1, yshift=1cm] {Remove junk features};
\node (s2) [below of=c1, yshift=1cm] {(1206, 400)};
\node (c2) [below of=s2, yshift=1cm] {Remove highly-correlated};
\node (s3) [below of=c2, yshift=1cm] {(1206, 375)};
\node (c3) [below of=s3, yshift=1cm] {Remove constant features};
\node (s4) [below of=c3, yshift=1cm] {(1206, 369)};
\node (c4) [below of=s4, yshift=1cm] {Remove subjects with all missing value};
\node (s5) [below of=c4, yshift=1cm] {(1113, 369)};
\node (c5) [below of=s5, yshift=1cm] {Remove sex and age class as target features};
\node (s6) [below of=c5, yshift=1cm] {(1113, 367)};

\draw [arrow] (start) -- (s1);
\draw [arrow] (s1) -- (c1);
\draw [arrow] (c1) -- (s2);
\draw [arrow] (s2) -- (c2);
\draw [arrow] (c2) -- (s3);
\draw [arrow] (s3) -- (c3);
\draw [arrow] (c3) -- (s4);
\draw [arrow] (s4) -- (c4);
\draw [arrow] (c4) -- (s5);
\draw [arrow] (s5) -- (c5);
\draw [arrow] (c5) -- (s6);
\end{tikzpicture}

\centering
\normalsize
\caption{Junk features include, book-keeping (IDs, scan counts) and subscale
items, or features with 200 or more NaN values. Highly-correlated features
include e.g\@. \texttt{various dexterity\_unadj} and
\texttt{dexterity\_ageadj} pairs, with correlations above 0.95, and other
features with correlations of 1.0.}
\end{figure}


% \begin{verbatim}
% #lst:cleaning .txt
% Behavioural Data initial shape
% 						⬇︎
% 				  (1206, 582)
% 						⬇︎
% Remove obvious junk, book-keeping (IDs, scan counts) and subscale items, or features with 200 or more NaN values
% 						⬇︎
% 				  (1206, 400)
% 						⬇︎
% Remove highly-correlated features (e.g. various dexterity_unadj
% and dexterity_ageadj pairs), with correlations above 0.95, and
% other features with correlations of 1.0.
% 						⬇︎
% 				  (1206, 375)
% 						⬇︎
% Remove constant and string columns
% 						⬇︎
% 				  (1206, 369)
% 						⬇︎
% Remove subjects with all missing values for all remaining features
% 						⬇︎
% 				  (1113, 369)
% 						⬇︎
% Remove sex and age class as potential target features
% 						⬇︎
% 				  (1113, 367)
% \end{verbatim}

Then, the remaining behavioural features were reduced in an exploratory
fashion.

First, clustering was performed on the features using HDBSCAN [cite
paper, cite sklearn algorithm] with the absolute value of the Pearson
correlation as the feature distance metric. HDBSCAN does not require
specifying the number of clusters, and allows for assignment to "noise" or
"background" clusters, and thus acts as a natural method to notice patterns
of correlation in the data. That is, by pooling together all features and
ignoring the feature measurement procedures, we can find natural clusters of measurements
without assuming that items are correlated simply by virtue of being members
of the same behavioural test or measurement instrument.

Second, after interpolating missing values with the feature means, HDBSCAN
feature clusters were reduced to a single dimension via factor analysis
[cite]. Factor analysis is a linear dimensionality reduction method wherein
the reduced factor can be interpreted as a latent variable that
well-describes the unreduced data, accounting for noise. FA was chosen here
over PCA as the reduced unidimensional feature is more readily interpreted as
a latent variable free from error variance [cite some good links
\href{https://en.wikipedia.org/w/index.php?title=Factor\_analysis&oldid=1223130166#Exploratory\_factor\_analysis\_(EFA)\_versus\_principal\_components\_analysis\_(PCA)}{here
} re: FA] and thus more suitable as a synthetic regression target.

The resulting synthetic feature names, clusters and loadings are shown in
Appendix B, as are descriptive statistics in
@Tbl:target\_stats. Most of the clusters appear to have face value. For
example, the factor we have labeled "int\_g\_like" involves a number of
features measuring fluid and crystalized intelligence, processing speed, and
performance at various sorting and reading tasks (Appendix B
@tbl:int\_g\_like), and so might be interpreted broadly as a general
intelligence factor [cite].

\begin{table}
\centering
\begin{tabular}{lrrrrrrrrr}
	\toprule
	target & mean & std & min & 2.5\% & 25\% & 50\% & 75\% & 97.5\% & max \\
	\midrule
	gambling\_perf       & -0.000 & 1.000 & -4.081 & -1.892 & -0.494 & -0.000 & 0.403 & 2.298 & 4.118 \\
	emotion\_perf        &  0.000 & 1.000 & -0.722 & -0.722 & -0.318 & -0.237 & 0.166 & 1.538 & 15.583\\
	language\_rt         &  0.000 & 1.000 & -3.813 & -1.684 & -0.711 & -0.012 & 0.589 & 2.220 & 4.465 \\
	relational\_rt       & -0.000 & 1.000 & -3.059 & -1.963 & -0.678 &  0.000 & 0.621 & 2.108 & 3.664 \\
	emotion\_rt          &  0.000 & 1.000 & -2.398 & -1.660 & -0.697 & -0.046 & 0.546 & 2.232 & 5.756 \\
	language\_perf       & -0.000 & 0.982 & -1.680 & -1.680 & -0.644 &  0.000 & 0.655 & 2.025 & 4.199 \\
	p\_matrices          & -0.000 & 1.000 & -1.491 & -1.364 & -0.837 & -0.097 & 0.625 & 2.067 & 4.982 \\
	social\_rt           &  0.000 & 0.983 & -1.861 & -1.455 & -0.717 & -0.113 & 0.480 & 2.407 & 4.406 \\
	psqi\_latent         &  0.000 & 0.998 & -1.734 & -1.377 & -0.652 & -0.288 & 0.438 & 2.251 & 5.138 \\
	gambling\_rt         & -0.000 & 1.000 & -1.888 & -1.431 & -0.753 & -0.165 & 0.618 & 2.228 & 6.054 \\
	social\_random\_perf & -0.000 & 0.999 & -0.861 & -0.787 & -0.770 & -0.759 & 0.276 & 2.367 & 4.485 \\
	int\_g\_like         &  0.000 & 1.000 & -2.190 & -2.180 & -0.700 &  0.074 & 0.681 & 1.861 & 2.589 \\
	neg\_emotionality    &  0.000 & 0.966 & -2.694 & -1.719 & -0.624 & -0.032 & 0.590 & 2.004 & 4.182 \\
	wm\_rt               &  0.000 & 1.000 & -2.504 & -1.674 & -0.678 & -0.015 & 0.564 & 2.286 & 5.335 \\
	wm\_perf             &  0.000 & 0.989 & -1.418 & -1.221 & -0.737 & -0.237 & 0.430 & 2.434 & 4.697 \\
	\bottomrule
\end{tabular}
\label{tab:target-stats}
\footnotesize
\caption{Percentiles and extrema for synthetic targets. Columns with percentages
indicate percentiles. All factors have a mean of approximately 0.0, and
standard deviation of approximately 1.0, by construction.}
\normalsize
\end{table}


\paragraph{FreeSurfer Comparisons}

As CMC features are derived from features extracted from FreeSurfer (FS)
cortical analyses, we compare model predictions on feature sets that: include
only FS features, include only CMC features, and include both FS and CMC
features.

\paragraph{AutoML via \texttt{df-analyze}}

\textcolor{orange}{Get Derek to provide a little description of the psychological explained
variance test with citation(s) was performed. Anything interesting from the
machine learning analysis? Can you run this again with current df-analyze? If
so is there anything interesting in there worth including in this manuscript?
If so describe the methods here and adapt the text immediately below. If not,
cut the text below.}

Predictive analyses were completed with
\href{github.com/stfxecutables/df-analyze}{\texttt{df-analyze}}, publicly
available AutoML [cite] software, developed in house to automate data
cleaning and preprocessing, and the subsequent feature selection, fitting,
tuning, and validation of various classic machine-learning (ML) models from
scikit-learn [cite], LightGBM [cite], and PyTorch [cite]. The software was
previously applied to brain MRI predictive application focused on
schizophrenia diagnostics \citep{levmanMorphologicalStudySchizophrenia2022}.

In the current work, \texttt{df-analyze} is used to compare all combinations
of the following:

\begin{itemize}
\item feature sets (FS only, CMC only, and FS and CMC features)
\item regression models (ElasticNet [cite], LightGBM [cite], and a dummy regressor)
\item feature selection methods
	\begin{itemize}
		\item no selection (all features are used for predictions)
		\item stepwise (forward stepwise selection with a linear model)
		\item embedded methods (feature importances from ElasticNet and LightGBM)
		\item two filter methods:
		\begin{enumerate}
			\item mutual information of each univariate feature with respect to the synthetic target
			\item predictive accuracy of each univariate feature with respect to the synthetic target
		\end{enumerate}
	\end{itemize}
\item synthetic targets (see [Appendix B](appendix-b))
\end{itemize}

As all synthetic targets are continuous variables, we use \(R^2\), the
coefficient of determination, and the mean absolute error (MAE) to assess
model fit. For any combination of feature set, feature selection method, and
target, a dummy regression model (which predicts either the target mean or
median, whichever results in better cross-validated performance) is fit, with
the intention that prediction performance of other models is considered
meaningful only if both \(R^2 > 0\) and the the non-dummy model has a lower MAE
than the dummy model.

All models are hyperparameter-tuned with Optuna [cite] using a tuning budget of
100 trials and 5-fold cross-validation for evaluation. All tuning, and feature
selection is done on a separate training data split using 50\% of the available
samples, with final results reported on the remaining 50\%, to ensure there is
no double-dipping or data leakage during these steps.



\textcolor{orange}{phenotypic features from the patients in our dataset from
the proposed CMC biomarkers. The targeted phenotypic features include insert
list and basic descriptions and justification for their selection - input
from Derek. The software selects the best combination of CMC biomarkers, the
best combination of traditional FreeSurfer biomarkers, and the best
combination of both, while being restricted by the user to limit the analysis
to a fixed number of biomarkers double check this with Derek. The number of
fixed biomarkers to provide to machine learning is varied, and the resultant
predictive capacity of the models, alongside which biomarkers were selected
for, is reported. This was just a blurb of what we could/maybe do, any
\texttt{df-analyze} methods providing something interesting to say is fine,
please adapt the text. Making \texttt{df-analyze}
\href{github.com/stfxecutables/df-analyze}{publicly available} facilitates
reproducibility of our study findings, alongside our use of large public
datasets of MRI examinations. In addition, we make available source code used
for the analyses in this study
\href{https://github.com/stfxecutables/cortical_congruence}{online} [TODO:
get a DOI for this and make public] analytic software that helps demonstrate
to the user how to use df-analyze and how to perform the statistical analyses
employed in this study.}


\section{Results}


\textcolor{orange}{\subsection{[DEREK] Note}}

\textcolor{orange}{There are generally no differences for group comparisons
(male vs.\ female, left vs.\ right), and also each CMC feature "class" has 34
ROIs, which is too much to summarize in almost any kind of plot. Tables are
probably best?}


\textcolor{orange}{Violion / box /
\href{https://seaborn.pydata.org/generated/seaborn.boxenplot.html}{boxen
plots} will all end up looking like @Fig:cmc\_sex\_diffs below. These plots
are actually sorted so that the most significant differences are at the left,
and still, nothing is really visually discernably as different between cases,
and this is so with swarm, strip, scatter, violin, box, or boxen plots.}

\subsection{Descriptive Statistic}

\subsubsection{CMC Features by Sex}

While all pseudo-volume features differ significantly by sex, otherwise,
only some lateral and bilateral CMC metrics differ significantly based on
sex, with the majority of these differences being such that females having
slightly higher congruence (Table~\ref{tab:sig-cmc-sex}).


\begin{table}
\centering
\begin{tabular}{lrrr}
	\toprule
	ROI & d & U & U (p) \\
	\midrule
	bh-temporalpole & -0.294 & 127593.000 & 0.000 \\
	lh-temporalpole & -0.283 & 129332.000 & 0.000 \\
	bh-insula       &  0.233 & 174258.000 & 0.004 \\
	rh-temporalpole & -0.229 & 134217.000 & 0.010 \\
	bh-paracentral  & -0.187 & 136173.000 & 0.035 \\
	rh-paracentral  & -0.177 & 136298.000 & 0.039 \\
	\bottomrule
\end{tabular}
\label{tab:sig-cmc-sex}
\footnotesize
\caption{CMC features with significant separation by sex.\
d = Cohen's d, with positive sign indicating larger metric values for males;
U = Mann-Whitney U, U (p) = p-value for Mann-Whitney U.
bh = bilateral (both hemispheres) CMC metric;
lh/rh = left/right hemisphere lateral CMC metric
Note: p-values are adjusted for multiple comparisons using the
Holm-Bonferroni stepdown method.}
\normalsize
\end{table}

For CMC feature classes, both pseudo-volume and left lateral CMC features
differ significantly in the overall spread standard deviation of their mean values
(\ref{tab:sig-cmc-scale-sex}). That is, male CMC pseudo-volume and left lateral
CMC features have more variable average values for males, relative to
females, though this difference is quite small, practically, and remains
significant only for pseudo-volume features when considering a robust measure
of scale (IQR).

\begin{table}
\centering
\begin{tabular}{lrrrrrr}
	\toprule
	CMC class & \(d_{\sigma}\) & \(d_\text{IQR}\)  & \(w_{\sigma}\) & \(w_\text{IQR}\)   & \(p_{\sigma}\)   & \(p_\text{IQR}\) \\
	\midrule
	Pseudo-volume        &  0.145 &  0.130 &  25.000 &  42.000 & 0.000 & 0.000 \\
	Left Lateral         &  0.021 &  0.029 & 146.000 & 184.000 & 0.009 & 0.053 \\
	Asym (signed ratio)  & -0.041 &  0.019 & 204.000 & 232.000 & 0.113 & 0.270 \\
	Bilateral            &  0.012 & -0.001 & 204.000 & 290.000 & 0.113 & 0.906 \\
	Asym (signed diff)   & -0.010 &  0.007 & 252.000 & 289.000 & 0.447 & 0.893 \\
	Asym (unsigned diff) & -0.002 &  0.001 & 287.000 & 289.000 & 0.866 & 0.893 \\
	Right Lateral        & -0.002 & -0.007 & 293.000 & 267.000 & 0.946 & 0.612 \\
	\bottomrule
\end{tabular}
\footnotesize
\caption{CMC feature classes with significant differences in scale, by sex.
\(\sigma\) = standard deviation; IQR = interquartile range;
\(d_{[\cdot]}\) = Cohen's \(d\) for metric \([\cdot]\), with positive sign indicating larger metric values for males;
\(W_{[\cdot]}\) = Wilcoxon signed rank test \(W\) for measure of scale \([\cdot]\);
\(p_{[\cdot]}\) = p-value for \(W_{[\cdot]}\);
Note: p-values are adjusted for multiple comparisons using the Holm-Bonferroni stepdown method}
\normalsize
\label{tab:sig-cmc-scale-sex}
\end{table}


\subsubsection{CMC Features by Laterality}

We find the lateral CMC measures differ significantly for all but the
superior frontal and parsopercularis regions (\ref{tab:lateral-cmc}).

\begin{table}
\centering
\begin{tabular}{lrrrrr}
	\toprule
	ROI & d &  \(W\) & \(W_{\sigma}\) & \(W_{\text{IQR}}\) & \(p\) \\
	\midrule
	inferiortemporal         & -1.890 &  0.000      & 254.000 & 272.000 & 0.000 \\
	insula                   & -1.778 &  21.000     & 254.000 & 272.000 & 0.000 \\
	pericalcarine            & -1.883 &  0.000      & 254.000 & 272.000 & 0.000 \\
	precuneus                & -1.772 &  0.000      & 254.000 & 272.000 & 0.000 \\
	precentral               & -1.661 &  75.000     & 254.000 & 272.000 & 0.000 \\
	rostralanteriorcingulate & 1.673  &  143.000    & 254.000 & 272.000 & 0.000 \\
	fusiform                 & 1.706  &  260.000    & 254.000 & 272.000 & 0.000 \\
	superiorparietal         & -1.598 &  376.000    & 254.000 & 272.000 & 0.000 \\
	postcentral              & 1.625  &  534.000    & 254.000 & 272.000 & 0.000 \\
	paracentral              & -1.406 &  593.000    & 254.000 & 272.000 & 0.000 \\
	caudalanteriorcingulate  & 1.630  &  1092.000   & 254.000 & 272.000 & 0.000 \\
	lateralorbitofrontal     & -1.337 &  1710.000   & 254.000 & 272.000 & 0.000 \\
	isthmuscingulate         & -1.498 &  1886.000   & 254.000 & 272.000 & 0.000 \\
	cuneus                   & -1.648 &  2429.000   & 254.000 & 272.000 & 0.000 \\
	transversetemporal       & -1.510 &  2995.000   & 254.000 & 272.000 & 0.000 \\
	lingual                  &  1.531 &  5488.000   & 254.000 & 272.000 & 0.000 \\
	posteriorcingulate       &  1.375 &  8640.000   & 254.000 & 272.000 & 0.000 \\
	superiortemporal         &  1.176 &  18322.000  & 254.000 & 272.000 & 0.000 \\
	parahippocampal          & -0.934 &  39694.000  & 254.000 & 272.000 & 0.000 \\
	middletemporal           &  0.940 &  44794.000  & 254.000 & 272.000 & 0.000 \\
	supramarginal            &  1.019 &  59895.000  & 254.000 & 272.000 & 0.000 \\
	frontalpole              &  0.880 &  70533.000  & 254.000 & 272.000 & 0.000 \\
	parsorbitalis            & -0.939 &  76597.000  & 254.000 & 272.000 & 0.000 \\
	lateraloccipital         &  0.894 &  79894.000  & 254.000 & 272.000 & 0.000 \\
	parstriangularis         &  0.611 &  90160.000  & 254.000 & 272.000 & 0.000 \\
	caudalmiddlefrontal      &  0.589 &  127030.000 & 254.000 & 272.000 & 0.000 \\
	inferiorparietal         &  0.482 &  171122.000 & 254.000 & 272.000 & 0.000 \\
	medialorbitofrontal      & -0.357 &  180895.000 & 254.000 & 272.000 & 0.000 \\
	bankssts                 & -0.417 &  187242.000 & 254.000 & 272.000 & 0.000 \\
	entorhinal               & -0.364 &  192057.000 & 254.000 & 272.000 & 0.000 \\
	rostralmiddlefrontal     &  0.275 &  224623.000 & 254.000 & 272.000 & 0.000 \\
	temporalpole             & -0.216 &  235096.000 & 254.000 & 272.000 & 0.000 \\
	parsopercularis          & -0.055 &  298155.000 & 254.000 & 272.000 & 0.541 \\
	superiorfrontal          &  0.030 &  306615.000 & 254.000 & 272.000 & 0.754 \\
	\bottomrule
\end{tabular}
\footnotesize
\caption{Measures of separation of lateral CMC features (left vs.\ right
hemisphere). d = Cohen's d, with a positive sign indicating greater congruence in left hemisphere ROIs;
U = Mann-Whitney U, U (p) = p-value for Mann-Whitney U;
W = Wilcoxon signed rank test; W (p) = p-value for W
Note: p-values are adjusted for multiple comparisons using the Holm-Bonferroni stepdown method
}
\normalsize
\label{tab:lateral-cmc}
\end{table}


\subsubsection{CMC Features by Age and Sex}

Significant feature age correlations are shown in Table~\ref{tab:cmc-age-spearman}.

\begin{table}
\centering
\begin{tabular}{llrrrrrrr}
	\toprule
	ROI & CMC class & $r$ & $r_p$ & $r_M$ & $r_{M_p}$ & $r_F$ & $r_{F_p}$ & $p_{\min}$ \\
	\midrule
	bh-superiorfrontal & Bilateral & -0.1795 & 0.0000 & -0.1870 & 0.0198 & -0.1909 & 0.0020 & 0.0000 \\
	bh-caudalmiddlefrontal & Bilateral & -0.1778 & 0.0000 & -0.1848 & 0.0249 & -0.1649 & 0.0389 & 0.0000 \\
	lh-middletemporal & Pseudo-volume & -0.1709 & 0.0000 & -0.0418 & 1.0000 & -0.1476 & 0.2224 & 0.0000 \\
	bh-isthmuscingulate & Pseudo-volume & -0.1675 & 0.0000 & -0.0902 & 1.0000 & -0.0870 & 1.0000 & 0.0000 \\
	bh-middletemporal & Pseudo-volume & -0.1667 & 0.0000 & -0.0429 & 1.0000 & -0.1312 & 0.9631 & 0.0000 \\
	lh-isthmuscingulate & Pseudo-volume & -0.1656 & 0.0000 & -0.0960 & 1.0000 & -0.0799 & 1.0000 & 0.0000 \\
	lh-lateralorbitofrontal & Left Lateral & -0.1652 & 0.0000 & -0.1757 & 0.0602 & -0.1382 & 0.5262 & 0.0000 \\
	bh-lateralorbitofrontal & Pseudo-volume & -0.1651 & 0.0000 & 0.0139 & 1.0000 & -0.1747 & 0.0134 & 0.0000 \\
	rh-lateralorbitofrontal & Pseudo-volume & -0.1635 & 0.0000 & 0.0313 & 1.0000 & -0.1924 & 0.0016 & 0.0000 \\
	rh-superiorfrontal & Right Lateral & -0.1626 & 0.0000 & -0.1755 & 0.0609 & -0.1649 & 0.0391 & 0.0000 \\
	bh-postcentral & Pseudo-volume & -0.1611 & 0.0001 & -0.0072 & 1.0000 & -0.1353 & 0.6805 & 0.0001 \\
	rh-inferiorparietal & Pseudo-volume & -0.1574 & 0.0001 & -0.0610 & 1.0000 & -0.1198 & 1.0000 & 0.0001 \\
	rh-postcentral & Pseudo-volume & -0.1548 & 0.0002 & -0.0199 & 1.0000 & -0.1181 & 1.0000 & 0.0002 \\
	bh-lateralorbitofrontal & Bilateral & -0.1542 & 0.0002 & -0.1637 & 0.1791 & -0.1306 & 1.0000 & 0.0002 \\
	bh-rostralmiddlefrontal & Pseudo-volume & -0.1534 & 0.0002 & 0.0151 & 1.0000 & -0.1530 & 0.1321 & 0.0002 \\
	bh-inferiorparietal & Pseudo-volume & -0.1531 & 0.0003 & -0.0526 & 1.0000 & -0.1024 & 1.0000 & 0.0003 \\
	lh-superiorfrontal & Left Lateral & -0.1476 & 0.0007 & -0.1489 & 0.6230 & -0.1585 & 0.0761 & 0.0007 \\
	lh-postcentral & Pseudo-volume & -0.1475 & 0.0007 & 0.0114 & 1.0000 & -0.1421 & 0.3679 & 0.0007 \\
	rh-caudalmiddlefrontal & Right Lateral & -0.1467 & 0.0008 & -0.1574 & 0.3076 & -0.1451 & 0.2807 & 0.0008 \\
	lh-lateralorbitofrontal & Pseudo-volume & -0.1460 & 0.0009 & -0.0012 & 1.0000 & -0.1333 & 0.8081 & 0.0009 \\
	rh-middletemporal & Pseudo-volume & -0.1455 & 0.0010 & -0.0368 & 1.0000 & -0.1070 & 1.0000 & 0.0010 \\
	rh-rostralmiddlefrontal & Pseudo-volume & -0.1438 & 0.0013 & 0.0228 & 1.0000 & -0.1489 & 0.1977 & 0.0013 \\
	bh-inferiortemporal & Pseudo-volume & -0.1433 & 0.0014 & -0.0436 & 1.0000 & -0.0866 & 1.0000 & 0.0014 \\
	lh-caudalmiddlefrontal & Left Lateral & -0.1406 & 0.0022 & -0.1354 & 1.0000 & -0.1313 & 0.9563 & 0.0022 \\
	lh-inferiortemporal & Pseudo-volume & -0.1398 & 0.0025 & -0.0372 & 1.0000 & -0.0951 & 1.0000 & 0.0025 \\
	bh-fusiform & Pseudo-volume & -0.1391 & 0.0028 & -0.0177 & 1.0000 & -0.0807 & 1.0000 & 0.0028 \\
	lh-fusiform & Pseudo-volume & -0.1386 & 0.0031 & -0.0473 & 1.0000 & -0.0780 & 1.0000 & 0.0031 \\
	bh-superiorparietal & Pseudo-volume & -0.1354 & 0.0051 & -0.0110 & 1.0000 & -0.1461 & 0.2545 & 0.0051 \\
	rh-isthmuscingulate & Pseudo-volume & -0.1347 & 0.0058 & -0.0322 & 1.0000 & -0.0956 & 1.0000 & 0.0058 \\
	rh-superiorfrontal & Pseudo-volume & -0.1343 & 0.0061 & -0.0112 & 1.0000 & -0.0915 & 1.0000 & 0.0061 \\
	lh-rostralmiddlefrontal & Pseudo-volume & -0.1334 & 0.0070 & 0.0071 & 1.0000 & -0.1302 & 1.0000 & 0.0070 \\
	bh-superiortemporal & Pseudo-volume & -0.1334 & 0.0071 & -0.0164 & 1.0000 & -0.0794 & 1.0000 & 0.0071 \\
	bh-superiorfrontal & Pseudo-volume & -0.1317 & 0.0092 & 0.0030 & 1.0000 & -0.0904 & 1.0000 & 0.0092 \\
	rh-superiorparietal & Pseudo-volume & -0.1304 & 0.0113 & -0.0136 & 1.0000 & -0.1196 & 1.0000 & 0.0113 \\
	bh-frontalpole & Pseudo-volume & -0.1296 & 0.0127 & -0.0467 & 1.0000 & -0.1093 & 1.0000 & 0.0127 \\
	bh-precentral & Bilateral & -0.1292 & 0.0135 & -0.1227 & 1.0000 & -0.1452 & 0.2787 & 0.0135 \\
	rh-superiortemporal & Pseudo-volume & -0.1289 & 0.0141 & -0.0312 & 1.0000 & -0.0820 & 1.0000 & 0.0141 \\
	bh-parstriangularis & Pseudo-volume & -0.1287 & 0.0146 & -0.0280 & 1.0000 & -0.0761 & 1.0000 & 0.0146 \\
	bh-rostralmiddlefrontal & Bilateral & -0.1253 & 0.0242 & -0.1491 & 0.6121 & -0.0928 & 1.0000 & 0.0242 \\
	rh-posteriorcingulate & Pseudo-volume & -0.1248 & 0.0259 & -0.0042 & 1.0000 & -0.1350 & 0.6939 & 0.0259 \\
	rh-parstriangularis & Right Lateral & -0.1068 & 0.2954 & -0.1839 & 0.0271 & -0.0426 & 1.0000 & 0.0271 \\
	bh-precentral & Pseudo-volume & -0.1242 & 0.0282 & 0.0236 & 1.0000 & -0.0647 & 1.0000 & 0.0282 \\
	lh-posteriorcingulate & Left Lateral & -0.1241 & 0.0286 & -0.1145 & 1.0000 & -0.1163 & 1.0000 & 0.0286 \\
	lh-parsopercularis & Pseudo-volume & -0.1238 & 0.0300 & -0.0737 & 1.0000 & -0.0705 & 1.0000 & 0.0300 \\
	bh-precuneus & Pseudo-volume & -0.1237 & 0.0304 & -0.0031 & 1.0000 & -0.1123 & 1.0000 & 0.0304 \\
	rh-parstriangularis & Pseudo-volume & -0.1226 & 0.0357 & -0.0316 & 1.0000 & -0.0848 & 1.0000 & 0.0357 \\
	bh-frontalpole & Bilateral & -0.1223 & 0.0370 & -0.0943 & 1.0000 & -0.1600 & 0.0655 & 0.0370 \\
	rh-frontalpole & Pseudo-volume & -0.1223 & 0.0373 & -0.0728 & 1.0000 & -0.0873 & 1.0000 & 0.0373 \\
	bh-parsopercularis & Pseudo-volume & -0.1223 & 0.0373 & -0.0648 & 1.0000 & -0.0699 & 1.0000 & 0.0373 \\
	rh-precuneus & Pseudo-volume & -0.1214 & 0.0423 & -0.0022 & 1.0000 & -0.0939 & 1.0000 & 0.0423 \\
	% lh-superiorfrontal & Pseudo-volume & -0.1180 & 0.0684 & 0.0127 & 1.0000 & -0.0856 & 1.0000 & 0.0684 \\
	% lh-precentral & Pseudo-volume & -0.1165 & 0.0837 & 0.0320 & 1.0000 & -0.0709 & 1.0000 & 0.0837 \\
	% rh-fusiform & Pseudo-volume & -0.1159 & 0.0913 & 0.0233 & 1.0000 & -0.0650 & 1.0000 & 0.0913 \\
	% bh-bankssts & Pseudo-volume & -0.1157 & 0.0937 & -0.0361 & 1.0000 & -0.0396 & 1.0000 & 0.0937 \\
	% bh-posteriorcingulate & Bilateral & -0.1150 & 0.1035 & -0.1309 & 1.0000 & -0.0897 & 1.0000 & 0.1035 \\
	% bh-parsorbitalis & Pseudo-volume & -0.1147 & 0.1069 & -0.0021 & 1.0000 & -0.1067 & 1.0000 & 0.1069 \\
	% lh-precuneus & Pseudo-volume & -0.1140 & 0.1177 & -0.0031 & 1.0000 & -0.1093 & 1.0000 & 0.1177 \\
	% bh-rostralanteriorcingulate & Bilateral & -0.1139 & 0.1195 & -0.0863 & 1.0000 & -0.1232 & 1.0000 & 0.1195 \\
	% rh-precentral & Right Lateral & -0.1137 & 0.1226 & -0.0785 & 1.0000 & -0.1487 & 0.1997 & 0.1226 \\
	% bh-rostralanteriorcingulate & Pseudo-volume & -0.1135 & 0.1262 & -0.0011 & 1.0000 & -0.0512 & 1.0000 & 0.1262 \\
	% lh-inferiorparietal & Pseudo-volume & -0.1124 & 0.1457 & -0.0389 & 1.0000 & -0.0522 & 1.0000 & 0.1457 \\
	% rh-caudalmiddlefrontal & Pseudo-volume & -0.1120 & 0.1537 & -0.0575 & 1.0000 & -0.0425 & 1.0000 & 0.1537 \\
	% lh-superiorparietal & Pseudo-volume & -0.1118 & 0.1572 & -0.0079 & 1.0000 & -0.1240 & 1.0000 & 0.1572 \\
	% lh-frontalpole & Left Lateral & -0.1111 & 0.1715 & -0.1376 & 1.0000 & -0.1031 & 1.0000 & 0.1715 \\
	% lh-precentral & Left Lateral & -0.1110 & 0.1739 & -0.1309 & 1.0000 & -0.1061 & 1.0000 & 0.1739 \\
	% lh-rostralanteriorcingulate & Pseudo-volume & -0.1108 & 0.1785 & 0.0198 & 1.0000 & -0.0863 & 1.0000 & 0.1785 \\
	% lh-superiortemporal & Pseudo-volume & -0.1098 & 0.2037 & 0.0071 & 1.0000 & -0.0571 & 1.0000 & 0.2037 \\
	% rh-precentral & Pseudo-volume & -0.1094 & 0.2129 & 0.0196 & 1.0000 & -0.0406 & 1.0000 & 0.2129 \\
	% rh-lateralorbitofrontal & Right Lateral & -0.1090 & 0.2254 & -0.1175 & 1.0000 & -0.0936 & 1.0000 & 0.2254 \\
	% bh-caudalmiddlefrontal & Pseudo-volume & -0.1086 & 0.2369 & -0.0369 & 1.0000 & -0.0469 & 1.0000 & 0.2369 \\
	% bh-posteriorcingulate & Pseudo-volume & -0.1081 & 0.2509 & 0.0221 & 1.0000 & -0.0995 & 1.0000 & 0.2509 \\
	% bh-medialorbitofrontal & Pseudo-volume & -0.1081 & 0.2519 & 0.0262 & 1.0000 & -0.0703 & 1.0000 & 0.2519 \\
	% lh-parsorbitalis & Pseudo-volume & -0.1077 & 0.2630 & -0.0204 & 1.0000 & -0.0687 & 1.0000 & 0.2630 \\
	% lh-parstriangularis & Pseudo-volume & -0.1074 & 0.2727 & -0.0229 & 1.0000 & -0.0606 & 1.0000 & 0.2727 \\
	% rh-inferiortemporal & Pseudo-volume & -0.1071 & 0.2828 & -0.0454 & 1.0000 & -0.0432 & 1.0000 & 0.2828 \\
	% bh-parstriangularis & Bilateral & -0.1037 & 0.4342 & -0.1567 & 0.3260 & -0.0673 & 1.0000 & 0.3260 \\
	% rh-medialorbitofrontal & Pseudo-volume & -0.1058 & 0.3322 & 0.0523 & 1.0000 & -0.0960 & 1.0000 & 0.3322 \\
	% bh-lateraloccipital & Pseudo-volume & -0.1044 & 0.3958 & 0.0266 & 1.0000 & -0.0776 & 1.0000 & 0.3958 \\
	% rh-lingual & Pseudo-volume & -0.1043 & 0.4045 & 0.0285 & 1.0000 & -0.1094 & 1.0000 & 0.4045 \\
	% rh-bankssts & Pseudo-volume & -0.1036 & 0.4378 & -0.0758 & 1.0000 & -0.0046 & 1.0000 & 0.4378 \\
	% rh-frontalpole & Right Lateral & -0.0929 & 1.0000 & -0.0491 & 1.0000 & -0.1401 & 0.4419 & 0.4419 \\
	% lh-supramarginal & Pseudo-volume & -0.1031 & 0.4668 & 0.0047 & 1.0000 & -0.0526 & 1.0000 & 0.4668 \\
	% rh-rostralmiddlefrontal & Right Lateral & -0.1009 & 0.6105 & -0.1305 & 1.0000 & -0.0769 & 1.0000 & 0.6105 \\
	% lh-rostralmiddlefrontal & Left Lateral & -0.0999 & 0.6829 & -0.1180 & 1.0000 & -0.0661 & 1.0000 & 0.6829 \\
	% rh-parsopercularis & Pseudo-volume & -0.0999 & 0.6829 & -0.0461 & 1.0000 & -0.0595 & 1.0000 & 0.6829 \\
	% rh-middletemporal & Right Lateral & -0.0998 & 0.6883 & -0.0807 & 1.0000 & -0.0941 & 1.0000 & 0.6883 \\
	% lh-parsopercularis & Left Lateral & -0.0984 & 0.8089 & -0.1292 & 1.0000 & -0.0654 & 1.0000 & 0.8089 \\
	% lh-frontalpole & Pseudo-volume & -0.0976 & 0.8886 & -0.0086 & 1.0000 & -0.0887 & 1.0000 & 0.8886 \\
	% lh-superiortemporal & Left Lateral & -0.0972 & 0.9400 & -0.1300 & 1.0000 & -0.0680 & 1.0000 & 0.9400 \\
	% bh-middletemporal & Bilateral & -0.0969 & 0.9699 & -0.0865 & 1.0000 & -0.0904 & 1.0000 & 0.9699 \\
	% bh-transversetemporal & Asym (unsigned diff) & -0.0200 & 1.0000 & 0.0252 & 1.0000 & -0.0594 & 1.0000 & 1.0000 \\
	% bh-superiortemporal & Asym (signed ratio) & -0.0184 & 1.0000 & -0.0267 & 1.0000 & -0.0265 & 1.0000 & 1.0000 \\
	% bh-caudalmiddlefrontal & Asym (signed ratio) & -0.0267 & 1.0000 & -0.0045 & 1.0000 & -0.0239 & 1.0000 & 1.0000 \\
	% bh-cuneus & Asym (signed ratio) & -0.0134 & 1.0000 & -0.0461 & 1.0000 & 0.0171 & 1.0000 & 1.0000 \\
	% bh-bankssts & Asym (signed ratio) & 0.0141 & 1.0000 & 0.0272 & 1.0000 & 0.0330 & 1.0000 & 1.0000 \\
	% bh-superiorparietal & Asym (signed ratio) & 0.0526 & 1.0000 & -0.0022 & 1.0000 & 0.0797 & 1.0000 & 1.0000 \\
	% bh-superiorfrontal & Asym (signed ratio) & -0.0198 & 1.0000 & -0.0184 & 1.0000 & -0.0256 & 1.0000 & 1.0000 \\
	% bh-temporalpole & Asym (unsigned diff) & -0.0356 & 1.0000 & -0.0084 & 1.0000 & -0.0602 & 1.0000 & 1.0000 \\
	% bh-temporalpole & Asym (signed ratio) & -0.0000 & 1.0000 & 0.0412 & 1.0000 & -0.0314 & 1.0000 & 1.0000 \\
	% bh-transversetemporal & Asym (signed ratio) & 0.0272 & 1.0000 & -0.0230 & 1.0000 & 0.0653 & 1.0000 & 1.0000 \\
	% bh-caudalanteriorcingulate & Asym (signed ratio) & 0.0236 & 1.0000 & -0.0148 & 1.0000 & 0.0770 & 1.0000 & 1.0000 \\
	% bh-rostralmiddlefrontal & Asym (signed ratio) & -0.0105 & 1.0000 & -0.0260 & 1.0000 & 0.0291 & 1.0000 & 1.0000 \\
	% bh-entorhinal & Asym (signed ratio) & 0.0538 & 1.0000 & 0.0159 & 1.0000 & 0.0897 & 1.0000 & 1.0000 \\
	% bh-supramarginal & Asym (signed ratio) & 0.0187 & 1.0000 & 0.0367 & 1.0000 & 0.0172 & 1.0000 & 1.0000 \\
	% bh-parahippocampal & Asym (signed ratio) & 0.0135 & 1.0000 & 0.0296 & 1.0000 & 0.0099 & 1.0000 & 1.0000 \\
	% bh-fusiform & Asym (signed ratio) & 0.0327 & 1.0000 & 0.0549 & 1.0000 & -0.0127 & 1.0000 & 1.0000 \\
	% bh-middletemporal & Asym (signed ratio) & 0.0459 & 1.0000 & 0.0209 & 1.0000 & 0.0454 & 1.0000 & 1.0000 \\
	% bh-parsopercularis & Asym (signed ratio) & -0.0359 & 1.0000 & -0.0463 & 1.0000 & -0.0129 & 1.0000 & 1.0000 \\
	% bh-medialorbitofrontal & Asym (signed ratio) & 0.0172 & 1.0000 & 0.0039 & 1.0000 & 0.0283 & 1.0000 & 1.0000 \\
	% bh-lingual & Asym (signed ratio) & -0.0163 & 1.0000 & 0.0180 & 1.0000 & -0.0311 & 1.0000 & 1.0000 \\
	% bh-lateralorbitofrontal & Asym (signed ratio) & -0.0401 & 1.0000 & -0.0635 & 1.0000 & -0.0155 & 1.0000 & 1.0000 \\
	% bh-parsorbitalis & Asym (signed ratio) & -0.0221 & 1.0000 & -0.0248 & 1.0000 & -0.0219 & 1.0000 & 1.0000 \\
	% bh-parstriangularis & Asym (signed ratio) & -0.0049 & 1.0000 & 0.0265 & 1.0000 & -0.0437 & 1.0000 & 1.0000 \\
	% bh-pericalcarine & Asym (signed ratio) & -0.0210 & 1.0000 & -0.0742 & 1.0000 & 0.0177 & 1.0000 & 1.0000 \\
	% bh-postcentral & Asym (signed ratio) & -0.0463 & 1.0000 & -0.0905 & 1.0000 & -0.0147 & 1.0000 & 1.0000 \\
	% bh-posteriorcingulate & Asym (signed ratio) & -0.0424 & 1.0000 & -0.0332 & 1.0000 & -0.0518 & 1.0000 & 1.0000 \\
	% bh-precentral & Asym (signed ratio) & 0.0102 & 1.0000 & -0.0569 & 1.0000 & 0.0598 & 1.0000 & 1.0000 \\
	% bh-precuneus & Asym (signed ratio) & 0.0043 & 1.0000 & -0.0179 & 1.0000 & 0.0201 & 1.0000 & 1.0000 \\
	% bh-lateraloccipital & Asym (signed ratio) & 0.0278 & 1.0000 & -0.0309 & 1.0000 & 0.0747 & 1.0000 & 1.0000 \\
	% bh-rostralanteriorcingulate & Asym (signed ratio) & -0.0196 & 1.0000 & -0.0591 & 1.0000 & -0.0139 & 1.0000 & 1.0000 \\
	% bh-isthmuscingulate & Asym (signed ratio) & 0.0487 & 1.0000 & 0.0263 & 1.0000 & 0.0557 & 1.0000 & 1.0000 \\
	% bh-insula & Asym (signed ratio) & -0.0247 & 1.0000 & -0.0507 & 1.0000 & 0.0150 & 1.0000 & 1.0000 \\
	% bh-inferiortemporal & Asym (signed ratio) & 0.0292 & 1.0000 & -0.0136 & 1.0000 & 0.0622 & 1.0000 & 1.0000 \\
	% bh-inferiorparietal & Asym (signed ratio) & -0.0114 & 1.0000 & -0.0154 & 1.0000 & -0.0074 & 1.0000 & 1.0000 \\
	% bh-paracentral & Asym (signed ratio) & -0.0003 & 1.0000 & -0.0336 & 1.0000 & 0.0289 & 1.0000 & 1.0000 \\
	% bh-frontalpole & Asym (signed ratio) & -0.0107 & 1.0000 & -0.0819 & 1.0000 & 0.0531 & 1.0000 & 1.0000 \\
	% lh-bankssts & Left Lateral & -0.0484 & 1.0000 & -0.0448 & 1.0000 & -0.0248 & 1.0000 & 1.0000 \\
	% lh-entorhinal & Pseudo-volume & -0.0695 & 1.0000 & 0.0124 & 1.0000 & 0.0151 & 1.0000 & 1.0000 \\
	% bh-cuneus & Pseudo-volume & -0.0676 & 1.0000 & 0.0441 & 1.0000 & -0.0577 & 1.0000 & 1.0000 \\
	% lh-pericalcarine & Pseudo-volume & -0.0288 & 1.0000 & 0.1137 & 1.0000 & -0.0457 & 1.0000 & 1.0000 \\
	% lh-posteriorcingulate & Pseudo-volume & -0.0563 & 1.0000 & 0.0432 & 1.0000 & -0.0214 & 1.0000 & 1.0000 \\
	% lh-temporalpole & Pseudo-volume & -0.0448 & 1.0000 & 0.0511 & 1.0000 & 0.0090 & 1.0000 & 1.0000 \\
	% lh-transversetemporal & Pseudo-volume & -0.0473 & 1.0000 & 0.0507 & 1.0000 & -0.0294 & 1.0000 & 1.0000 \\
	% rh-caudalanteriorcingulate & Pseudo-volume & -0.0586 & 1.0000 & -0.0285 & 1.0000 & -0.0084 & 1.0000 & 1.0000 \\
	% rh-cuneus & Pseudo-volume & -0.0645 & 1.0000 & 0.0037 & 1.0000 & -0.0166 & 1.0000 & 1.0000 \\
	% rh-entorhinal & Pseudo-volume & -0.0370 & 1.0000 & 0.0113 & 1.0000 & 0.0612 & 1.0000 & 1.0000 \\
	% rh-insula & Pseudo-volume & -0.0674 & 1.0000 & 0.0040 & 1.0000 & -0.0008 & 1.0000 & 1.0000 \\
	% rh-lateraloccipital & Pseudo-volume & -0.0887 & 1.0000 & 0.0026 & 1.0000 & -0.0309 & 1.0000 & 1.0000 \\
	% rh-paracentral & Pseudo-volume & -0.0691 & 1.0000 & 0.0287 & 1.0000 & -0.0221 & 1.0000 & 1.0000 \\
	% rh-parahippocampal & Pseudo-volume & -0.0426 & 1.0000 & 0.0973 & 1.0000 & -0.0342 & 1.0000 & 1.0000 \\
	% rh-parsorbitalis & Pseudo-volume & -0.0938 & 1.0000 & -0.0038 & 1.0000 & -0.0977 & 1.0000 & 1.0000 \\
	% rh-pericalcarine & Pseudo-volume & -0.0627 & 1.0000 & 0.0471 & 1.0000 & -0.0362 & 1.0000 & 1.0000 \\
	% rh-rostralanteriorcingulate & Pseudo-volume & -0.0859 & 1.0000 & -0.0082 & 1.0000 & -0.0173 & 1.0000 & 1.0000 \\
	% rh-supramarginal & Pseudo-volume & -0.0360 & 1.0000 & 0.0717 & 1.0000 & -0.0041 & 1.0000 & 1.0000 \\
	% lh-parahippocampal & Pseudo-volume & -0.0485 & 1.0000 & 0.0284 & 1.0000 & -0.0069 & 1.0000 & 1.0000 \\
	% lh-paracentral & Pseudo-volume & -0.0434 & 1.0000 & 0.0675 & 1.0000 & -0.0217 & 1.0000 & 1.0000 \\
	% lh-medialorbitofrontal & Pseudo-volume & -0.0865 & 1.0000 & 0.0068 & 1.0000 & -0.0281 & 1.0000 & 1.0000 \\
	% lh-lingual & Pseudo-volume & -0.0435 & 1.0000 & 0.0905 & 1.0000 & -0.0232 & 1.0000 & 1.0000 \\
	% bh-entorhinal & Pseudo-volume & -0.0602 & 1.0000 & 0.0066 & 1.0000 & 0.0538 & 1.0000 & 1.0000 \\
	% bh-insula & Pseudo-volume & -0.0758 & 1.0000 & 0.0288 & 1.0000 & -0.0315 & 1.0000 & 1.0000 \\
	% bh-lingual & Pseudo-volume & -0.0905 & 1.0000 & 0.0556 & 1.0000 & -0.0816 & 1.0000 & 1.0000 \\
	% bh-paracentral & Pseudo-volume & -0.0609 & 1.0000 & 0.0549 & 1.0000 & -0.0190 & 1.0000 & 1.0000 \\
	% bh-parahippocampal & Pseudo-volume & -0.0491 & 1.0000 & 0.0601 & 1.0000 & -0.0148 & 1.0000 & 1.0000 \\
	% bh-pericalcarine & Pseudo-volume & -0.0436 & 1.0000 & 0.1030 & 1.0000 & -0.0431 & 1.0000 & 1.0000 \\
	% bh-supramarginal & Pseudo-volume & -0.0786 & 1.0000 & 0.0428 & 1.0000 & -0.0243 & 1.0000 & 1.0000 \\
	% bh-caudalanteriorcingulate & Pseudo-volume & -0.0740 & 1.0000 & -0.0303 & 1.0000 & -0.0158 & 1.0000 & 1.0000 \\
	% bh-temporalpole & Pseudo-volume & -0.0516 & 1.0000 & 0.0216 & 1.0000 & 0.0111 & 1.0000 & 1.0000 \\
	% lh-bankssts & Pseudo-volume & -0.0889 & 1.0000 & 0.0100 & 1.0000 & -0.0458 & 1.0000 & 1.0000 \\
	% lh-caudalanteriorcingulate & Pseudo-volume & -0.0603 & 1.0000 & -0.0006 & 1.0000 & -0.0211 & 1.0000 & 1.0000 \\
	% lh-caudalmiddlefrontal & Pseudo-volume & -0.0801 & 1.0000 & -0.0003 & 1.0000 & -0.0379 & 1.0000 & 1.0000 \\
	% lh-cuneus & Pseudo-volume & -0.0548 & 1.0000 & 0.0345 & 1.0000 & -0.0555 & 1.0000 & 1.0000 \\
	% bh-supramarginal & Asym (unsigned diff) & 0.0054 & 1.0000 & 0.0256 & 1.0000 & 0.0085 & 1.0000 & 1.0000 \\
	% lh-insula & Pseudo-volume & -0.0643 & 1.0000 & 0.0332 & 1.0000 & -0.0474 & 1.0000 & 1.0000 \\
	% lh-lateraloccipital & Pseudo-volume & -0.0822 & 1.0000 & 0.0458 & 1.0000 & -0.0862 & 1.0000 & 1.0000 \\
	% bh-transversetemporal & Pseudo-volume & -0.0513 & 1.0000 & 0.0370 & 1.0000 & -0.0176 & 1.0000 & 1.0000 \\
	% bh-superiortemporal & Asym (unsigned diff) & -0.0202 & 1.0000 & -0.0425 & 1.0000 & -0.0183 & 1.0000 & 1.0000 \\
	% bh-paracentral & Asym (unsigned diff) & -0.0020 & 1.0000 & 0.0222 & 1.0000 & -0.0284 & 1.0000 & 1.0000 \\
	% bh-superiorfrontal & Asym (unsigned diff) & -0.0338 & 1.0000 & -0.0365 & 1.0000 & -0.0412 & 1.0000 & 1.0000 \\
	% rh-medialorbitofrontal & Right Lateral & -0.0894 & 1.0000 & -0.0605 & 1.0000 & -0.0985 & 1.0000 & 1.0000 \\
	% rh-paracentral & Right Lateral & -0.0391 & 1.0000 & -0.0744 & 1.0000 & -0.0440 & 1.0000 & 1.0000 \\
	% rh-parahippocampal & Right Lateral & -0.0145 & 1.0000 & -0.0922 & 1.0000 & 0.0140 & 1.0000 & 1.0000 \\
	% rh-parsopercularis & Right Lateral & -0.0514 & 1.0000 & -0.0831 & 1.0000 & -0.0354 & 1.0000 & 1.0000 \\
	% rh-parsorbitalis & Right Lateral & -0.0100 & 1.0000 & -0.0353 & 1.0000 & 0.0008 & 1.0000 & 1.0000 \\
	% rh-pericalcarine & Right Lateral & -0.0252 & 1.0000 & -0.0010 & 1.0000 & -0.0347 & 1.0000 & 1.0000 \\
	% rh-postcentral & Right Lateral & -0.0119 & 1.0000 & -0.0068 & 1.0000 & -0.0313 & 1.0000 & 1.0000 \\
	% rh-posteriorcingulate & Right Lateral & -0.0746 & 1.0000 & -0.0979 & 1.0000 & -0.0426 & 1.0000 & 1.0000 \\
	% rh-precuneus & Right Lateral & -0.0623 & 1.0000 & -0.0673 & 1.0000 & -0.0669 & 1.0000 & 1.0000 \\
	% rh-rostralanteriorcingulate & Right Lateral & -0.0805 & 1.0000 & -0.0514 & 1.0000 & -0.0755 & 1.0000 & 1.0000 \\
	% rh-superiorparietal & Right Lateral & -0.0601 & 1.0000 & -0.0551 & 1.0000 & -0.0522 & 1.0000 & 1.0000 \\
	% rh-superiortemporal & Right Lateral & -0.0386 & 1.0000 & -0.0581 & 1.0000 & -0.0075 & 1.0000 & 1.0000 \\
	% rh-supramarginal & Right Lateral & -0.0494 & 1.0000 & -0.0790 & 1.0000 & -0.0328 & 1.0000 & 1.0000 \\
	% rh-temporalpole & Right Lateral & 0.0189 & 1.0000 & -0.0427 & 1.0000 & 0.0258 & 1.0000 & 1.0000 \\
	% rh-transversetemporal & Right Lateral & -0.0166 & 1.0000 & 0.0199 & 1.0000 & -0.0504 & 1.0000 & 1.0000 \\
	% bh-bankssts & Bilateral & -0.0628 & 1.0000 & -0.0631 & 1.0000 & -0.0570 & 1.0000 & 1.0000 \\
	% bh-caudalanteriorcingulate & Bilateral & -0.0798 & 1.0000 & -0.0834 & 1.0000 & -0.0885 & 1.0000 & 1.0000 \\
	% bh-parsorbitalis & Bilateral & -0.0404 & 1.0000 & -0.0845 & 1.0000 & -0.0186 & 1.0000 & 1.0000 \\
	% bh-parsopercularis & Bilateral & -0.0954 & 1.0000 & -0.1344 & 1.0000 & -0.0633 & 1.0000 & 1.0000 \\
	% bh-parahippocampal & Bilateral & 0.0006 & 1.0000 & -0.0584 & 1.0000 & 0.0176 & 1.0000 & 1.0000 \\
	% bh-paracentral & Bilateral & -0.0551 & 1.0000 & -0.0916 & 1.0000 & -0.0610 & 1.0000 & 1.0000 \\
	% bh-medialorbitofrontal & Bilateral & -0.0755 & 1.0000 & -0.0618 & 1.0000 & -0.0719 & 1.0000 & 1.0000 \\
	% bh-lingual & Bilateral & -0.0194 & 1.0000 & -0.0467 & 1.0000 & 0.0073 & 1.0000 & 1.0000 \\
	% rh-lingual & Right Lateral & -0.0079 & 1.0000 & -0.0377 & 1.0000 & 0.0121 & 1.0000 & 1.0000 \\
	% bh-lateraloccipital & Bilateral & 0.0227 & 1.0000 & 0.0063 & 1.0000 & 0.0661 & 1.0000 & 1.0000 \\
	% bh-insula & Bilateral & -0.0350 & 1.0000 & 0.0148 & 1.0000 & -0.0225 & 1.0000 & 1.0000 \\
	% bh-inferiortemporal & Bilateral & -0.0647 & 1.0000 & -0.0737 & 1.0000 & -0.0645 & 1.0000 & 1.0000 \\
	% bh-inferiorparietal & Bilateral & -0.0684 & 1.0000 & -0.0683 & 1.0000 & -0.0677 & 1.0000 & 1.0000 \\
	% bh-fusiform & Bilateral & -0.0367 & 1.0000 & -0.0139 & 1.0000 & -0.0605 & 1.0000 & 1.0000 \\
	% bh-entorhinal & Bilateral & -0.0271 & 1.0000 & -0.0551 & 1.0000 & -0.0300 & 1.0000 & 1.0000 \\
	% bh-cuneus & Bilateral & -0.0280 & 1.0000 & -0.0504 & 1.0000 & -0.0081 & 1.0000 & 1.0000 \\
	% bh-isthmuscingulate & Bilateral & -0.0334 & 1.0000 & -0.0205 & 1.0000 & -0.0665 & 1.0000 & 1.0000 \\
	% bh-pericalcarine & Bilateral & -0.0484 & 1.0000 & -0.0880 & 1.0000 & 0.0001 & 1.0000 & 1.0000 \\
	% rh-lateraloccipital & Right Lateral & -0.0087 & 1.0000 & 0.0178 & 1.0000 & -0.0031 & 1.0000 & 1.0000 \\
	% rh-insula & Right Lateral & -0.0014 & 1.0000 & 0.0586 & 1.0000 & -0.0320 & 1.0000 & 1.0000 \\
	% lh-caudalanteriorcingulate & Left Lateral & -0.0468 & 1.0000 & -0.0668 & 1.0000 & -0.0260 & 1.0000 & 1.0000 \\
	% lh-cuneus & Left Lateral & -0.0264 & 1.0000 & -0.0576 & 1.0000 & 0.0010 & 1.0000 & 1.0000 \\
	% lh-entorhinal & Left Lateral & -0.0057 & 1.0000 & -0.0316 & 1.0000 & -0.0022 & 1.0000 & 1.0000 \\
	% lh-fusiform & Left Lateral & -0.0080 & 1.0000 & 0.0169 & 1.0000 & -0.0479 & 1.0000 & 1.0000 \\
	% lh-inferiorparietal & Left Lateral & -0.0622 & 1.0000 & -0.0520 & 1.0000 & -0.0692 & 1.0000 & 1.0000 \\
	% lh-inferiortemporal & Left Lateral & -0.0440 & 1.0000 & -0.0650 & 1.0000 & -0.0325 & 1.0000 & 1.0000 \\
	% lh-insula & Left Lateral & -0.0389 & 1.0000 & -0.0205 & 1.0000 & -0.0068 & 1.0000 & 1.0000 \\
	% lh-isthmuscingulate & Left Lateral & -0.0155 & 1.0000 & -0.0133 & 1.0000 & -0.0323 & 1.0000 & 1.0000 \\
	% lh-lateraloccipital & Left Lateral & 0.0251 & 1.0000 & -0.0200 & 1.0000 & 0.0786 & 1.0000 & 1.0000 \\
	% lh-lingual & Left Lateral & -0.0485 & 1.0000 & -0.0665 & 1.0000 & -0.0191 & 1.0000 & 1.0000 \\
	% lh-medialorbitofrontal & Left Lateral & -0.0526 & 1.0000 & -0.0508 & 1.0000 & -0.0366 & 1.0000 & 1.0000 \\
	% lh-middletemporal & Left Lateral & -0.0587 & 1.0000 & -0.0530 & 1.0000 & -0.0554 & 1.0000 & 1.0000 \\
	% lh-paracentral & Left Lateral & -0.0487 & 1.0000 & -0.0911 & 1.0000 & -0.0512 & 1.0000 & 1.0000 \\
	% lh-parahippocampal & Left Lateral & 0.0058 & 1.0000 & -0.0379 & 1.0000 & 0.0207 & 1.0000 & 1.0000 \\
	% lh-parsorbitalis & Left Lateral & -0.0562 & 1.0000 & -0.0946 & 1.0000 & -0.0427 & 1.0000 & 1.0000 \\
	% lh-parstriangularis & Left Lateral & -0.0819 & 1.0000 & -0.1044 & 1.0000 & -0.0767 & 1.0000 & 1.0000 \\
	% lh-pericalcarine & Left Lateral & -0.0446 & 1.0000 & -0.0925 & 1.0000 & -0.0008 & 1.0000 & 1.0000 \\
	% rh-inferiortemporal & Right Lateral & -0.0814 & 1.0000 & -0.0503 & 1.0000 & -0.1184 & 1.0000 & 1.0000 \\
	% rh-inferiorparietal & Right Lateral & -0.0529 & 1.0000 & -0.0535 & 1.0000 & -0.0468 & 1.0000 & 1.0000 \\
	% rh-fusiform & Right Lateral & -0.0512 & 1.0000 & -0.0344 & 1.0000 & -0.0589 & 1.0000 & 1.0000 \\
	% rh-entorhinal & Right Lateral & -0.0419 & 1.0000 & -0.0482 & 1.0000 & -0.0614 & 1.0000 & 1.0000 \\
	% rh-cuneus & Right Lateral & -0.0205 & 1.0000 & 0.0046 & 1.0000 & -0.0462 & 1.0000 & 1.0000 \\
	% rh-caudalanteriorcingulate & Right Lateral & -0.0706 & 1.0000 & -0.0577 & 1.0000 & -0.1002 & 1.0000 & 1.0000 \\
	% rh-isthmuscingulate & Right Lateral & -0.0871 & 1.0000 & -0.0761 & 1.0000 & -0.0950 & 1.0000 & 1.0000 \\
	% rh-bankssts & Right Lateral & -0.0531 & 1.0000 & -0.0480 & 1.0000 & -0.0663 & 1.0000 & 1.0000 \\
	% lh-temporalpole & Left Lateral & 0.0172 & 1.0000 & 0.0044 & 1.0000 & -0.0215 & 1.0000 & 1.0000 \\
	% lh-supramarginal & Left Lateral & -0.0591 & 1.0000 & -0.0920 & 1.0000 & -0.0276 & 1.0000 & 1.0000 \\
	% lh-superiorparietal & Left Lateral & 0.0296 & 1.0000 & -0.0234 & 1.0000 & 0.0550 & 1.0000 & 1.0000 \\
	% lh-rostralanteriorcingulate & Left Lateral & -0.0910 & 1.0000 & -0.0878 & 1.0000 & -0.0939 & 1.0000 & 1.0000 \\
	% lh-precuneus & Left Lateral & -0.0379 & 1.0000 & -0.0526 & 1.0000 & -0.0274 & 1.0000 & 1.0000 \\
	% lh-postcentral & Left Lateral & -0.0562 & 1.0000 & -0.0862 & 1.0000 & -0.0515 & 1.0000 & 1.0000 \\
	% lh-transversetemporal & Left Lateral & 0.0312 & 1.0000 & 0.0015 & 1.0000 & 0.0473 & 1.0000 & 1.0000 \\
	% bh-superiorparietal & Asym (unsigned diff) & -0.0602 & 1.0000 & -0.0094 & 1.0000 & -0.0859 & 1.0000 & 1.0000 \\
	% bh-postcentral & Bilateral & -0.0346 & 1.0000 & -0.0465 & 1.0000 & -0.0454 & 1.0000 & 1.0000 \\
	% bh-superiorparietal & Bilateral & 0.0069 & 1.0000 & -0.0427 & 1.0000 & 0.0340 & 1.0000 & 1.0000 \\
	% bh-supramarginal & Asym (signed diff) & 0.0146 & 1.0000 & 0.0274 & 1.0000 & 0.0165 & 1.0000 & 1.0000 \\
	% bh-temporalpole & Asym (signed diff) & 0.0004 & 1.0000 & 0.0446 & 1.0000 & -0.0311 & 1.0000 & 1.0000 \\
	% bh-transversetemporal & Asym (signed diff) & 0.0245 & 1.0000 & -0.0216 & 1.0000 & 0.0633 & 1.0000 & 1.0000 \\
	% bh-bankssts & Asym (unsigned diff) & 0.0259 & 1.0000 & 0.0368 & 1.0000 & 0.0114 & 1.0000 & 1.0000 \\
	% bh-caudalanteriorcingulate & Asym (unsigned diff) & 0.0018 & 1.0000 & -0.0346 & 1.0000 & 0.0491 & 1.0000 & 1.0000 \\
	% bh-caudalmiddlefrontal & Asym (unsigned diff) & -0.0378 & 1.0000 & -0.0050 & 1.0000 & -0.0656 & 1.0000 & 1.0000 \\
	% bh-cuneus & Asym (unsigned diff) & 0.0081 & 1.0000 & 0.0406 & 1.0000 & -0.0225 & 1.0000 & 1.0000 \\
	% bh-entorhinal & Asym (unsigned diff) & 0.0058 & 1.0000 & 0.0047 & 1.0000 & -0.0142 & 1.0000 & 1.0000 \\
	% bh-frontalpole & Asym (unsigned diff) & -0.0146 & 1.0000 & -0.0774 & 1.0000 & 0.0311 & 1.0000 & 1.0000 \\
	% bh-fusiform & Asym (unsigned diff) & 0.0259 & 1.0000 & 0.0493 & 1.0000 & -0.0225 & 1.0000 & 1.0000 \\
	% bh-inferiorparietal & Asym (unsigned diff) & -0.0237 & 1.0000 & 0.0137 & 1.0000 & -0.0480 & 1.0000 & 1.0000 \\
	% bh-inferiortemporal & Asym (unsigned diff) & -0.0546 & 1.0000 & -0.0087 & 1.0000 & -0.0980 & 1.0000 & 1.0000 \\
	% bh-insula & Asym (unsigned diff) & 0.0175 & 1.0000 & 0.0488 & 1.0000 & -0.0162 & 1.0000 & 1.0000 \\
	% bh-isthmuscingulate & Asym (unsigned diff) & -0.0638 & 1.0000 & -0.0445 & 1.0000 & -0.0701 & 1.0000 & 1.0000 \\
	% bh-lateraloccipital & Asym (unsigned diff) & 0.0178 & 1.0000 & -0.0323 & 1.0000 & 0.0622 & 1.0000 & 1.0000 \\
	% bh-lateralorbitofrontal & Asym (unsigned diff) & 0.0195 & 1.0000 & 0.0394 & 1.0000 & -0.0010 & 1.0000 & 1.0000 \\
	% bh-lingual & Asym (unsigned diff) & -0.0208 & 1.0000 & 0.0111 & 1.0000 & -0.0335 & 1.0000 & 1.0000 \\
	% bh-rostralmiddlefrontal & Asym (unsigned diff) & -0.0796 & 1.0000 & -0.1133 & 1.0000 & -0.0450 & 1.0000 & 1.0000 \\
	% bh-rostralanteriorcingulate & Asym (unsigned diff) & -0.0345 & 1.0000 & -0.0720 & 1.0000 & -0.0241 & 1.0000 & 1.0000 \\
	% bh-precuneus & Asym (unsigned diff) & -0.0229 & 1.0000 & -0.0032 & 1.0000 & -0.0391 & 1.0000 & 1.0000 \\
	% bh-precentral & Asym (unsigned diff) & -0.0322 & 1.0000 & 0.0345 & 1.0000 & -0.0827 & 1.0000 & 1.0000 \\
	% bh-posteriorcingulate & Asym (unsigned diff) & -0.0570 & 1.0000 & -0.0537 & 1.0000 & -0.0571 & 1.0000 & 1.0000 \\
	% bh-postcentral & Asym (unsigned diff) & -0.0565 & 1.0000 & -0.1036 & 1.0000 & -0.0279 & 1.0000 & 1.0000 \\
	% bh-superiortemporal & Asym (signed diff) & -0.0295 & 1.0000 & -0.0421 & 1.0000 & -0.0324 & 1.0000 & 1.0000 \\
	% bh-pericalcarine & Asym (unsigned diff) & 0.0008 & 1.0000 & 0.0521 & 1.0000 & -0.0335 & 1.0000 & 1.0000 \\
	% bh-parsorbitalis & Asym (unsigned diff) & 0.0146 & 1.0000 & 0.0323 & 1.0000 & 0.0019 & 1.0000 & 1.0000 \\
	% bh-parsopercularis & Asym (unsigned diff) & -0.0282 & 1.0000 & -0.0006 & 1.0000 & -0.0530 & 1.0000 & 1.0000 \\
	% bh-parahippocampal & Asym (unsigned diff) & -0.0119 & 1.0000 & -0.0542 & 1.0000 & 0.0051 & 1.0000 & 1.0000 \\
	% rh-temporalpole & Pseudo-volume & -0.0513 & 1.0000 & -0.0302 & 1.0000 & 0.0093 & 1.0000 & 1.0000 \\
	% bh-middletemporal & Asym (unsigned diff) & 0.0273 & 1.0000 & 0.0107 & 1.0000 & 0.0231 & 1.0000 & 1.0000 \\
	% bh-medialorbitofrontal & Asym (unsigned diff) & -0.0061 & 1.0000 & -0.0074 & 1.0000 & -0.0171 & 1.0000 & 1.0000 \\
	% bh-parstriangularis & Asym (unsigned diff) & -0.0095 & 1.0000 & -0.0202 & 1.0000 & -0.0117 & 1.0000 & 1.0000 \\
	% bh-precuneus & Bilateral & -0.0513 & 1.0000 & -0.0630 & 1.0000 & -0.0401 & 1.0000 & 1.0000 \\
	% bh-superiorparietal & Asym (signed diff) & 0.0601 & 1.0000 & 0.0094 & 1.0000 & 0.0853 & 1.0000 & 1.0000 \\
	% bh-rostralmiddlefrontal & Asym (signed diff) & -0.0152 & 1.0000 & -0.0313 & 1.0000 & 0.0267 & 1.0000 & 1.0000 \\
	% bh-superiortemporal & Bilateral & -0.0631 & 1.0000 & -0.0879 & 1.0000 & -0.0277 & 1.0000 & 1.0000 \\
	% bh-supramarginal & Bilateral & -0.0631 & 1.0000 & -0.0893 & 1.0000 & -0.0464 & 1.0000 & 1.0000 \\
	% bh-temporalpole & Bilateral & 0.0192 & 1.0000 & -0.0243 & 1.0000 & -0.0004 & 1.0000 & 1.0000 \\
	% bh-transversetemporal & Bilateral & 0.0133 & 1.0000 & 0.0173 & 1.0000 & 0.0061 & 1.0000 & 1.0000 \\
	% bh-bankssts & Asym (signed diff) & 0.0154 & 1.0000 & 0.0270 & 1.0000 & 0.0366 & 1.0000 & 1.0000 \\
	% bh-caudalanteriorcingulate & Asym (signed diff) & 0.0025 & 1.0000 & -0.0348 & 1.0000 & 0.0510 & 1.0000 & 1.0000 \\
	% bh-caudalmiddlefrontal & Asym (signed diff) & -0.0311 & 1.0000 & -0.0106 & 1.0000 & -0.0270 & 1.0000 & 1.0000 \\
	% bh-cuneus & Asym (signed diff) & -0.0082 & 1.0000 & -0.0418 & 1.0000 & 0.0220 & 1.0000 & 1.0000 \\
	% bh-entorhinal & Asym (signed diff) & 0.0547 & 1.0000 & 0.0182 & 1.0000 & 0.0919 & 1.0000 & 1.0000 \\
	% bh-frontalpole & Asym (signed diff) & -0.0212 & 1.0000 & -0.0903 & 1.0000 & 0.0397 & 1.0000 & 1.0000 \\
	% bh-fusiform & Asym (signed diff) & 0.0261 & 1.0000 & 0.0496 & 1.0000 & -0.0221 & 1.0000 & 1.0000 \\
	% bh-inferiorparietal & Asym (signed diff) & -0.0130 & 1.0000 & -0.0138 & 1.0000 & -0.0114 & 1.0000 & 1.0000 \\
	% bh-inferiortemporal & Asym (signed diff) & 0.0546 & 1.0000 & 0.0087 & 1.0000 & 0.0980 & 1.0000 & 1.0000 \\
	% bh-insula & Asym (signed diff) & -0.0175 & 1.0000 & -0.0488 & 1.0000 & 0.0160 & 1.0000 & 1.0000 \\
	% bh-isthmuscingulate & Asym (signed diff) & 0.0615 & 1.0000 & 0.0430 & 1.0000 & 0.0662 & 1.0000 & 1.0000 \\
	% bh-lateraloccipital & Asym (signed diff) & 0.0281 & 1.0000 & -0.0273 & 1.0000 & 0.0749 & 1.0000 & 1.0000 \\
	% bh-lateralorbitofrontal & Asym (signed diff) & -0.0203 & 1.0000 & -0.0393 & 1.0000 & -0.0011 & 1.0000 & 1.0000 \\
	% bh-rostralanteriorcingulate & Asym (signed diff) & -0.0349 & 1.0000 & -0.0731 & 1.0000 & -0.0241 & 1.0000 & 1.0000 \\
	% bh-precuneus & Asym (signed diff) & 0.0229 & 1.0000 & 0.0032 & 1.0000 & 0.0391 & 1.0000 & 1.0000 \\
	% bh-precentral & Asym (signed diff) & 0.0324 & 1.0000 & -0.0345 & 1.0000 & 0.0830 & 1.0000 & 1.0000 \\
	% bh-posteriorcingulate & Asym (signed diff) & -0.0614 & 1.0000 & -0.0514 & 1.0000 & -0.0679 & 1.0000 & 1.0000 \\
	% bh-postcentral & Asym (signed diff) & -0.0560 & 1.0000 & -0.1037 & 1.0000 & -0.0264 & 1.0000 & 1.0000 \\
	% bh-pericalcarine & Asym (signed diff) & -0.0008 & 1.0000 & -0.0521 & 1.0000 & 0.0335 & 1.0000 & 1.0000 \\
	% bh-superiorfrontal & Asym (signed diff) & -0.0192 & 1.0000 & -0.0157 & 1.0000 & -0.0265 & 1.0000 & 1.0000 \\
	% bh-parstriangularis & Asym (signed diff) & -0.0107 & 1.0000 & 0.0174 & 1.0000 & -0.0461 & 1.0000 & 1.0000 \\
	% bh-parsopercularis & Asym (signed diff) & -0.0385 & 1.0000 & -0.0493 & 1.0000 & -0.0152 & 1.0000 & 1.0000 \\
	% bh-parahippocampal & Asym (signed diff) & 0.0130 & 1.0000 & 0.0396 & 1.0000 & 0.0055 & 1.0000 & 1.0000 \\
	% bh-paracentral & Asym (signed diff) & 0.0015 & 1.0000 & -0.0225 & 1.0000 & 0.0273 & 1.0000 & 1.0000 \\
	% bh-middletemporal & Asym (signed diff) & 0.0375 & 1.0000 & 0.0138 & 1.0000 & 0.0351 & 1.0000 & 1.0000 \\
	% bh-medialorbitofrontal & Asym (signed diff) & 0.0191 & 1.0000 & 0.0039 & 1.0000 & 0.0316 & 1.0000 & 1.0000 \\
	% bh-lingual & Asym (signed diff) & -0.0239 & 1.0000 & 0.0057 & 1.0000 & -0.0336 & 1.0000 & 1.0000 \\
	% bh-parsorbitalis & Asym (signed diff) & -0.0138 & 1.0000 & -0.0098 & 1.0000 & -0.0173 & 1.0000 & 1.0000 \\
	% rh-transversetemporal & Pseudo-volume & -0.0218 & 1.0000 & 0.0303 & 1.0000 & 0.0158 & 1.0000 & 1.0000 \\
	\bottomrule
\end{tabular}
\footnotesize
\caption{Spearman correlations between CMC feature classes, age, and sex.
$r$ = Spearman's correlation with age, all subjects;
$r_X$ = male/female correlation for X = M/F, respectively;
$[]_p$ = two-sided p-value for metric [];
$p_{\min}$ = smallest p-value of each row;
Note: All p-values are adjusted for multiple comparisons using the Holm-Bonferroni stepdown method
}
\normalsize
\label{tab:cmc-age-spearman}
\end{table}


\subsection{Predictive Analyses}

\subsubsection{Predictive Performance}

A breakdown of the proportion of models exceeding dummy regressor
performance, for each group of target and feature sets, is presented below in
Table~\ref{tab:cmc-p-target-predictive} and Table~\ref{tab:cmc-p-predictive}.
More detailed breakdowns by model are included in [Appendix C](appendix-c),
in Table~\ref{tab:cmc-model-p-predictive}. Overall, the
\texttt{int\_g\_like}, \texttt{language\_perf}, \texttt{emotion\_rt}, and
\texttt{gambling\_rt}" synthetic targets were most consistently predictable
across the majority of runs.

% from summary.py::summarize_hcp_prediction_percents

\begin{table}
\centering
\begin{tabular}{lr}
	\toprule
	target & exceeds\_dummy \\
	\midrule
	int\_g\_like & 0.983 \\
	language\_perf & 0.650 \\
	emotion\_rt & 0.517 \\
	gambling\_rt & 0.417 \\
	p\_matrices & 0.367 \\
	wm\_perf & 0.167 \\
	wm\_rt & 0.167 \\
	gambling\_perf & 0.125 \\
	social\_rt & 0.083 \\
	language\_rt & 0.033 \\
	emotion\_perf & 0.000 \\
	neg\_emotionality & 0.000 \\
	psqi\_latent & 0.000 \\
	relational\_rt & 0.000 \\
	social\_random\_perf & 0.000 \\
	\bottomrule
\end{tabular}
\footnotesize
\caption{Proportion of model runs where predictions exceed dummy performance.\
exceeds\_dummy = proportion of runs with performance exceeding dummy models;}
\normalsize
\label{tab:cmc-p-target-predictive}
\end{table}


\begin{table}
\centering
\begin{tabular}{llr}
	\toprule
	target & feats & exceeds\_dummy \\
	\midrule
	emotion\_perf & CMC & 0.000 \\
	emotion\_perf & FS & 0.000 \\
	emotion\_perf & FS+CMC & 0.000 \\
	emotion\_rt & CMC & 0.417 \\
	emotion\_rt & FS & 0.583 \\
	emotion\_rt & FS+CMC & 0.583 \\
	gambling\_perf & CMC & 0.083 \\
	gambling\_perf & FS & 0.167 \\
	gambling\_rt & CMC & 0.500 \\
	gambling\_rt & FS & 0.333 \\
	gambling\_rt & FS+CMC & 0.417 \\
	int\_g\_like & CMC & 1.000 \\
	int\_g\_like & FS & 1.000 \\
	int\_g\_like & FS+CMC & 0.917 \\
	language\_perf & CMC & 0.667 \\
	language\_perf & FS & 0.667 \\
	language\_perf & FS+CMC & 0.583 \\
	language\_rt & CMC & 0.000 \\
	language\_rt & FS & 0.083 \\
	language\_rt & FS+CMC & 0.000 \\
	neg\_emotionality & CMC & 0.000 \\
	neg\_emotionality & FS & 0.000 \\
	neg\_emotionality & FS+CMC & 0.000 \\
	p\_matrices & CMC & 0.167 \\
	p\_matrices & FS & 0.583 \\
	p\_matrices & FS+CMC & 0.333 \\
	psqi\_latent & CMC & 0.000 \\
	psqi\_latent & FS & 0.000 \\
	relational\_rt & CMC & 0.000 \\
	relational\_rt & FS & 0.000 \\
	social\_random\_perf & CMC & 0.000 \\
	social\_random\_perf & FS & 0.000 \\
	social\_rt & CMC & 0.000 \\
	social\_rt & FS & 0.167 \\
	wm\_perf & CMC & 0.167 \\
	wm\_perf & FS & 0.250 \\
	wm\_perf & FS+CMC & 0.000 \\
	wm\_rt & CMC & 0.250 \\
	wm\_rt & FS & 0.083 \\
	wm\_rt & FS+CMC & 0.167 \\
	\bottomrule
\end{tabular}
\footnotesize
\caption{%
Performance of models on runs exceeding dummy model performance.\
feats = feature set;
exceeds\_dummy = proportion of runs with performance exceeding dummy models;
For target definitions, see [Appendix B](appendix-b).
}
\normalsize
\label{tab:cmc-p-predictive}
\end{table}



The observed predictive performances on the final holdout set are presented
in Table~\ref{tab:cmc-p-model-predictive}. For the \texttt{int\_g\_like} and \texttt{language\_perf}
synthetic target features, the holdout coefficient of determination
consistently exceeded 0.1, and, in many cases, for completely linear models
(ElasticNet). If we interpret the \(R^2\) value then as the proportion of
explained variance [cite], then this suggests either FS or CMC features (or
both) explain over 10\% of the variance in these two synthetic targets.

[**FOR DISCUSSION**] All synthetic targets that are best-predicted are
\texttt{int\_g\_like}, \texttt{language\_perf} (performance on language
tasks, e.g.\ verbal intelligence), \texttt{wm\_perf} (performance on working
memory tasks, often considered part of intelligence), and, maybe weakly,
\texttt{p\_matrices}, which is performance on Raven's progressive matrices,
often considered one of the simplest, cleanest correlates of the general
intelligence factor *g*. Granted, in almost all cases, just using FS features
will better predict these synthetic targets than CMC alone, or FS+CMC both,
which strongly suggests CMC features are poor transformations of FS features.
Nevertheless, this does suggest that these intelligence-related aspects are
most predictable from static morphology, as measured by FreeSurfer.




\subsubsection{Selected Features}

A number of model runs involve both FS and CMC feature sets. As \texttt{df-analyze}
automates simple feature selection via filter, wrapper, and embedded
methoods, it is possible to roughly examine the proportion of CMC features
selected by some of the most predictive models. These are show below in
Table~\ref{tab:cmc-feature-select}.

Note that performances are lower than in previous tables and figures, as the best
performing models often did not use a mixed feature set, or were those where no
feature selection was employed.

\begin{table}
\centering
\begin{tabular}{lllrrrrrr}
	\toprule
	target & selection & model & r2 & mae & mae+ & p\_sel\_cmc & p\_sel\_feat\_cmc & cmc\_win \\
	\midrule
	int\_g\_like & assoc & lgbm & 0.110 & 0.193 & 0.009 & 1.000 & 0.504 & True \\
	int\_g\_like & assoc & elastic & 0.072 & 0.192 & 0.010 & 1.000 & 0.504 & True \\
	p\_matrices & assoc & lgbm & 0.015 & 0.237 & 0.002 & 1.000 & 0.504 & True \\
	wm\_rt & assoc & lgbm & 0.016 & 0.197 & 0.000 & 1.000 & 0.504 & True \\
	gambling\_rt & assoc & lgbm & 0.019 & 0.203 & 0.003 & 1.000 & 0.504 & True \\
	p\_matrices & assoc & knn & 0.021 & 0.237 & 0.001 & 1.000 & 0.504 & True \\
	emotion\_rt & assoc & lgbm & 0.004 & 0.193 & 0.001 & 1.000 & 0.504 & True \\
	language\_perf & assoc & lgbm & 0.089 & 0.197 & 0.007 & 1.000 & 0.504 & True \\
	gambling\_rt & pred & lgbm & 0.021 & 0.200 & 0.006 & 0.467 & 0.469 & False \\
	language\_perf & embed\_lgbm & elastic & 0.073 & 0.201 & 0.003 & 0.088 & 0.455 & False \\
	language\_perf & embed\_lgbm & lgbm & 0.056 & 0.201 & 0.002 & 0.088 & 0.455 & False \\
	int\_g\_like & embed\_lgbm & elastic & 0.104 & 0.192 & 0.010 & 0.080 & 0.414 & False \\
	language\_perf & embed\_linear & lgbm & 0.090 & 0.197 & 0.007 & 0.410 & 0.386 & False \\
	emotion\_rt & pred & lgbm & 0.026 & 0.192 & 0.002 & 0.382 & 0.385 & False \\
	gambling\_rt & embed\_linear & lgbm & 0.018 & 0.202 & 0.003 & 0.386 & 0.375 & False \\
	int\_g\_like & embed\_linear & knn & 0.073 & 0.195 & 0.007 & 0.443 & 0.369 & False \\
	int\_g\_like & embed\_linear & elastic & 0.067 & 0.192 & 0.010 & 0.443 & 0.369 & False \\
	int\_g\_like & embed\_linear & lgbm & 0.071 & 0.196 & 0.006 & 0.443 & 0.369 & False \\
	emotion\_rt & embed\_lgbm & elastic & 0.026 & 0.193 & 0.001 & 0.022 & 0.367 & False \\
	emotion\_rt & embed\_linear & lgbm & 0.032 & 0.191 & 0.003 & 0.357 & 0.357 & False \\
	emotion\_rt & embed\_linear & knn & 0.012 & 0.194 & 0.000 & 0.357 & 0.357 & False \\
	int\_g\_like & pred & elastic & 0.076 & 0.191 & 0.011 & 0.331 & 0.333 & False \\
	int\_g\_like & pred & lgbm & 0.042 & 0.198 & 0.004 & 0.331 & 0.333 & False \\
	language\_perf & pred & lgbm & 0.087 & 0.197 & 0.007 & 0.324 & 0.325 & False \\
	\bottomrule
\end{tabular}
\footnotesize
\caption{Proportion of CMC features selected for best performing models;
p\_sel\_cmc = proportion of *all available* CMC features selected for final model;
p\_sel\_feat\_cmc = proportion of all *selected* features that are CMC features;
}
\normalsize
\label{tab:cmc-feature-select}
\end{table}


\section{Discussion}

The human brain's regional cortical development proceeds with a variety of
underlying factors maturing in tandem with one another. Cortical volume,
surface area and thickness are excellent examples of measurable biomarkers
that clearly exhibit interdependencies with one another. However, it should
be noted that the relative maturation of each of these biomarkers may proceed
at varying rates in any particular combination of brain region, patient or
pathology. Gray matter (GM) volume is known to increase with age, as does the
surface area, while cortical thickness thins with long-term development.
These three biomarkers are inter-related, and existing studies focused on
these measurements typically do not consider the inherent interdependence
between their respective development, even though underlying
interdependencies are inevitable. This paper presents a novel set of cortical
morphological congruence (CMC) biomarkers that are based on cortical volume,
surface area and thickness, \textcolor{red}{and produces measurements with
relatively small standard deviations (see Figure 1)}, implying potential
reliability and reproducibility from the proposed methods.

\subsection{Interpretation of CMC and its potential relationship with macro-structural cortical development}

The proposed CMC biomarkers, defined in equations 1 through 4, rely on
underlying measurements of gray matter volume (measured in mm3), surface area
(measured in mm2) and average cortical thickness (measured in mm). The nature
of the proposed equations are such that they each produce a unitless index of
CMC, different values of which imply potentially major differences in the
conformation of the local cortical region, potentially implying major
differences in the tissue's historical neurodevelopment. When a cortical
region exhibits a CMC (see equations 1 \& 2) measurement equal to 1, which is
expected for very regularly shaped structures, that region exhibits a
relatively simple cortical morphological presentation, and can be said to
have high underlying cortical morphological congruence. Examples of this can
be found consistently across patients in brain structures such as the banks
of the superior temporal sulcus (n=1,113, mean CMCLateral=1.0*, std dev
CMCLateral=0.025*), and the insula (mean CMCLateral=1.0?, std dev
CMCLateral=???). When CMC values deviate from 1, this implies varying degrees
of incongruence between the region's volumetric biomarkers and its surface
area and cortical thickness biomarkers combined. The directionality of that
incongruence (i.e.\ whether CMC is above or below 1) has major implications
for the presentation of the conformation of that tissue, and implies
differential cortical development has occurred.

When a cortical region exhibits CMC above 1, the GM volume has developed to
be larger than the surface area times the mean thickness. This can occur when
the overall growth of regional cortical tissue proceeds more quickly than
increases in the surface area. Broadly speaking, the morphological structure
that maximizes volume relative to surface area is the sphere. Thus, it is
expected that convex (and thus partly spherical) presentation on the surface
of the cortical region (well-rounded boundaries between the cortex and the
pia mater which surrounds the cortex) will contribute to CMC measurements
above 1. Regions such as the entorhinal cortex, which plays a role in working
memory and thus is a highly relied upon region of cortical tissue, exhibits
high CMC values (mean = 1.3**, standard deviation = 0.?). This implies that
the entorhinal region's development may have involved rapid increases in
volume relative to its respective increases in surface area. These high CMC
values may also implicate a distribution of pruning locations that supports
sulcal formation, leading to more convex (partly spherical) local surface
areas within the entorhinal cortex adjacent to locations of sulcal formation.
Thus, we hypothesize that pruning in the entorhinal cortex has been more
extensive (and possibly proceeded faster) than pruning in regions exhibiting
cortical morphological congruence (CMC = 1) such as the banks of the superior
temporal sulcus, or the insula. Results demonstrate that in addition to the
entorhinal cortex, multiple regions exhibit consistently high CMC values,
including the temporal pole, frontal pole, and pars orbitalis (see Figure 1).

When a cortical region exhibits CMC values below 1, the combination of the
surface area times the mean thickness has developed to be larger than the
gray matter volume. This can occur when the surface area, which is expected
to be affected by several underlying factors, including regional brain
growth, cortical folding and pruning, develops more rapidly than the growth
in overall regional gray matter volume alone. Additionally, the distribution
of locations of pruning within the cortex can result in the emergence of
comparatively complex surfaces relative to the more spherical/convex surfaces
already discussed, potentially resulting in comparatively large surface areas
yielding reduced values for our CMC biomarkers. Regions such as the
pericalcarine cortex exhibit low CMC values (mean = 0.95*, standard deviation
= \verb|??|), which could imply that surface area growth has outpaced corresponding
volumetric growth in this region's development relative to other cortical
regions.

\subsection{Potential for CMC to characterize important aspects of brain development}

The combination of regional cortical volume, surface area and mean thickness,
biomarkers with relatively high variability across patients, into a single
CMC biomarker with relatively small variability in regional cortical
measurements is noteworthy. Reliability and reproducibility are a major
ongoing challenge in neuroscience research
\citep{martinezReproducibilityBraincognitionRelationships2015,marekReproducibleBrainwideAssociation2022},
so any biomarkers that present consistently across a large population, and
reliably demonstrate differential presentation across cortical regions, has
considerable potential to assist in reliable and reproducible
characterization of the human brain.

For most of the CMC biomarkers evaluated, males and females exhibit highly
overlapping distributions, implying negligible differences in most cortical
regions, which could imply the proposed biomarkers provide standardization
benefits towards reproducible studies, and is consistent with largely
overlapping functional abilities between the genders in most capacities.
However, some male-female differences were observed in the temporal pole, the
frontal pole and the pars opercularis, with females exhibiting higher CMC
biomarkers on average than their male counterparts, though the distributions
are still overlapping (see Figure 1). The temporal pole has been implicated
in many functions, including emotional processing
\citep{corcoles-paradaFrontalInsularInput2019}, and the frontal pole has been
reported to contribute to control over emotional approach-avoidance actions
\citep{bramsonHumanLateralFrontal2020}. Thus, gender differences in the
presentation of the temporal and frontal poles, as assessed by CMC, may
assist in characterization of known gender differences in emotional
expression \citep{chaplinGenderEmotionExpression2015}. The pars opercularis
is involved in language processing \citep{greweEmergenceUnmarkedNew2005}, and
sex differences in the pars opercularis, as assessed with CMC, may be
indicative of underlying known differences in language development between
males and females \citep{satoNeurobiologySexDifferences2020}. Indeed, it is
encouraging that group-wise differences are observed in overlapping
distributions as although we know sex effects exist in emotional expression
and language development, there is a wide amount of variability in function
across both genders, which is reflected in our CMC biomarker results
exhibiting partially overlapping distributions between the sexes.


\textcolor{orange}{%
***Hopefully there is something related to phenotypic data that Derek's
analysis will uncover?
}

\textcolor{orange}{%
***Discuss variation explained psychological statistical test results here
}

\textcolor{orange}{%
***Discuss machine learning predictive capacity and features underlying those
predictions here
}

Although most cortical regions exhibited consistent CMC values in the left
and right hemispheres, we did observe asymmetries in the transverse temporal,
entorhinal, caudal anterior cingulate and pericalcarine regions. Asymmetries
have previously been observed in the entorhinal cortex
\citep{simicHemisphericAsymmetryModular2005}, with larger surface areas being
reported in the left hemisphere. This is consistent with our findings of
decreased CMC in the left hemisphere (increased surface area results in
decreased CMC). The transverse temporal cortex is known to exhibit leftward
asymmetries that are detectable by 31 weeks gestation
\citep{chiLeftRightAsymmetriesTemporal1977}, which is consistent with our
findings of decreased CMC in the left hemisphere. Asymmetries have also been
previously reported in the anterior cingulate
\citep{yanHemisphericAsymmetryCognitive2009}. The pericalcarine cortex has
also been reported to exhibit asymmetries
\citep{chiarelloStructuralAsymmetryHuman2016,koelkebeckContributionCorticalThickness2014},
which our analysis was also able to detect with CMC biomarkers. Our
identification of asymmetries of CMC biomarkers implies that our analyses
have considerable consistency with known asymmetric properties of the human
brain.

\subsection{Potential for CMC to characterize pathologies}

A wide variety of pathological conditions have been demonstrated to exhibit
abnormal phenotypic presentation of regions of the brain, including Down
Syndrome \citep{levmanStructuralMagneticResonance2019}, attention deficit
hyperactivity disorder (ADHD)
\citep{listonAtypicalPrefrontalConnectivity2011,stanleyEvidenceDevelopmentalAlterations2008},
schizophrenia
\citep{innocentiSchizophreniaNeurodevelopmentCorpus2003,keshavanSchizophreniaDueExcessive1994,feinbergCorticalPruningDevelopment1990,hoffmanCorticalPruningDevelopment1989,rimolCorticalThicknessSubcortical2010,narrMappingCorticalThickness2005,venkatasubramanianAutomatedMRIParcellation2008,vanharenChangesCorticalThickness2011,schultzReducedCorticalThickness2010,nesvagRegionalThinningCerebral2008,seitzAlterationGrayMatter2018,qiuHippocampalcorticalStructuralConnectivity2010,johnsonHippocampalShapeAbnormalities2013,mackinleyDeviantCorticalSulcation2020},
psychotic disorders \citep{bakkerCorticalMorphologyDifferences2016}, autism
\citep{khundrakpamCorticalThicknessAbnormalities2017,pereiraDifferencesCorticalStructure2018,zielinskiLongitudinalChangesCortical2014,levmanStructuralMagneticResonance2019,levmanClinicallyDetectableStructural2021},
and multiple sclerosis
\citep{brexLongitudinalStudyAbnormalities2002,losseffProgressiveCerebralAtrophy1996,chenRelatingNeocorticalPathology2004,sailerFocalThinningCerebral2003,levmanClinicallyDetectableStructural2021}.

Thus, future work will entail the characterization of the development of the
pathological brain with CMC\@. As an additional novel biomarker not previously
available, CMC may characterize regional abnormal development of the cortex
in a manner not previously characterized, and the feature measurements
generated by the approach outlined in this manuscript may also be a useful
addition to future machine learning / artificial intelligence technologies
that perform predictions for diagnostics, prognostics and treatment planning.
Future work will investigate the potential for a variety of pathologies to be
associated with macro-structural developmental abnormalities, such as
aberrant folding and sulcal formation, and thus CMC may assist in the
characterization of the macro-level phenotypic presentation of the brain. It
is hoped that the CMC technique presented in this manuscript will be helpful
in characterizing and understanding the developmental processes and
etiological factors associated with healthy brain development, as well as a
variety of neurodevelopmental disorders. It is also hoped that congruence
based biomarkers will assist in characterizing important aspects of healthy
and abnormal brain, reliably and reproducibly.

***

Talk to Bruce Fischl with Emi about possibility of it being included in future versions of FreeSurfer and whether he has any feedback that might be relevant and may want to join on this project as a co-author.


\section{Appendix A} \label{sec:appendix-a}


\begin{table}
\centering
\begin{tabular}{lrrr}
	\toprule
	ROI & d & W & W (p) \\
	\midrule
	ThickAvg\_\_inferiortemporal & 1.913 & 168552.000 & 0.000 \\
	ThickAvg\_\_rostralanteriorcingulate & -1.898 & 4993.500 & 0.000 \\
	ThickAvg\_\_fusiform & -1.843 & 0.000 & 0.000 \\
	ThickAvg\_\_pericalcarine & 1.842 & 7625.500 & 0.000 \\
	ThickAvg\_\_postcentral & -1.825 & 0.000 & 0.000 \\
	ThickAvg\_\_precentral & 1.824 & 0.000 & 0.000 \\
	ThickAvg\_\_isthmuscingulate & 1.780 & 0.000 & 0.000 \\
	ThickAvg\_\_insula & 1.774 & 0.000 & 0.000 \\
	ThickAvg\_\_paracentral & 1.751 & 30331.500 & 0.000 \\
	ThickAvg\_\_precuneus & 1.716 & 0.000 & 0.000 \\
	ThickAvg\_\_caudalanteriorcingulate & -1.651 & 0.000 & 0.000 \\
	ThickAvg\_\_cuneus & 1.607 & 0.000 & 0.000 \\
	ThickAvg\_\_lateralorbitofrontal & 1.599 & 0.000 & 0.000 \\
	ThickAvg\_\_lingual & -1.563 & 0.000 & 0.000 \\
	ThickAvg\_\_transversetemporal & 1.538 & 0.000 & 0.000 \\
	ThickAvg\_\_superiortemporal & -1.472 & 116444.500 & 0.000 \\
	ThickAvg\_\_posteriorcingulate & -1.470 & 0.000 & 0.000 \\
	ThickAvg\_\_parahippocampal & 1.466 & 0.000 & 0.000 \\
	ThickAvg\_\_superiorparietal & 1.433 & 0.000 & 0.000 \\
	Area\_\_frontalpole & -1.360 & 3510.500 & 0.000 \\
	SurfArea\_\_frontalpole & -1.360 & 47046.000 & 0.000 \\
	GrayVol\_\_frontalpole & -1.330 & 3527.500 & 0.000 \\
	SurfArea\_\_transversetemporal & 1.319 & 7695.000 & 0.000 \\
	Area\_\_transversetemporal & 1.319 & 12451.000 & 0.000 \\
	Area\_\_parsorbitalis & -1.266 & 4209.000 & 0.000 \\
	SurfArea\_\_parsorbitalis & -1.266 & 88337.000 & 0.000 \\
	GrayVol\_\_parsorbitalis & -1.265 & 2215.000 & 0.000 \\
	ThickAvg\_\_middletemporal & -1.218 & 0.000 & 0.000 \\
	GrayVol\_\_inferiorparietal & -1.163 & 2039.500 & 0.000 \\
	ThickAvg\_\_parstriangularis & -1.159 & 0.000 & 0.000 \\
	ThickAvg\_\_supramarginal & -1.123 & 0.000 & 0.000 \\
	GrayVol\_\_transversetemporal & 1.104 & 1464.500 & 0.000 \\
	SurfArea\_\_rostralanteriorcingulate & 1.084 & 0.000 & 0.000 \\
	Area\_\_rostralanteriorcingulate & 1.084 & 34500.000 & 0.000 \\
	Area\_\_inferiorparietal & -1.078 & 833.000 & 0.000 \\
	SurfArea\_\_inferiorparietal & -1.078 & 120187.500 & 0.000 \\
	ThickAvg\_\_frontalpole & -1.037 & 0.000 & 0.000 \\
	Thick\_\_\_medialorbitofrontal & -1.031 & 0.000 & 0.000 \\
	GrayVol\_\_rostralanteriorcingulate & 1.024 & 21433.500 & 0.000 \\
	ThickAvg\_\_lateraloccipital & -1.017 & 49.000 & 0.000 \\
	ThickAvg\_\_caudalmiddlefrontal & -0.973 & 11104.000 & 0.000 \\
	GrayVol\_\_parstriangularis & -0.966 & 41362.000 & 0.000 \\
	ThickAvg\_\_medialorbitofrontal & -0.955 & 0.000 & 0.000 \\
	SurfArea\_\_temporalpole & 0.929 & 91259.500 & 0.000 \\
	Area\_\_temporalpole & 0.929 & 187447.000 & 0.000 \\
	ThickAvg\_\_parsorbitalis & 0.915 & 0.000 & 0.000 \\
	Area\_\_parsopercularis & 0.911 & 41545.000 & 0.000 \\
	SurfArea\_\_parsopercularis & 0.911 & 297040.500 & 0.744 \\
	GrayVol\_\_parsopercularis & 0.896 & 38251.500 & 0.000 \\
	SurfArea\_\_parstriangularis & -0.867 & 16618.000 & 0.000 \\
	Area\_\_parstriangularis & -0.867 & 29132.500 & 0.000 \\
	SurfArea\_\_paracentral & -0.841 & 0.000 & 0.000 \\
	Area\_\_paracentral & -0.841 & 62069.500 & 0.000 \\
	GrayVol\_\_middletemporal & -0.803 & 18687.000 & 0.000 \\
	Thick\_\_\_bankssts & -0.777 & 0.000 & 0.000 \\
	ThickAvg\_\_temporalpole & -0.770 & 0.000 & 0.000 \\
	GrayVol\_\_paracentral & -0.766 & 43831.000 & 0.000 \\
	Thick\_\_\_caudalanteriorcingulate & 0.753 & 0.000 & 0.000 \\
	Thick\_\_\_temporalpole & -0.741 & 0.000 & 0.000 \\
	Area\_\_entorhinal & 0.738 & 126581.000 & 0.000 \\
	SurfArea\_\_entorhinal & 0.738 & 150876.000 & 0.000 \\
	Area\_\_middletemporal & -0.718 & 11374.500 & 0.000 \\
	SurfArea\_\_middletemporal & -0.718 & 12275.000 & 0.000 \\
	SurfArea\_\_caudalanteriorcingulate & -0.707 & 469.500 & 0.000 \\
	Area\_\_caudalanteriorcingulate & -0.707 & 151094.000 & 0.000 \\
	ThickAvg\_\_inferiorparietal & -0.706 & 226159.000 & 0.000 \\
	Thick\_\_\_inferiorparietal & -0.635 & 0.000 & 0.000 \\
	GrayVol\_\_pericalcarine & -0.593 & 57421.000 & 0.000 \\
	GrayVol\_\_caudalanteriorcingulate & -0.589 & 106594.500 & 0.000 \\
	Thick\_\_\_middletemporal & -0.555 & 0.000 & 0.000 \\
	SurfArea\_\_pericalcarine & -0.530 & 0.000 & 0.000 \\
	Area\_\_pericalcarine & -0.530 & 46557.500 & 0.000 \\
	SurfArea\_\_medialorbitofrontal & 0.496 & 22949.000 & 0.000 \\
	Area\_\_medialorbitofrontal & 0.496 & 285946.500 & 0.161 \\
	Thick\_\_\_parstriangularis & -0.486 & 8126.500 & 0.000 \\
	GrayVol\_\_entorhinal & 0.485 & 62042.000 & 0.000 \\
	Thick\_\_\_entorhinal & -0.484 & 0.000 & 0.000 \\
	ThickAvg\_\_entorhinal & -0.471 & 16634.500 & 0.000 \\
	Area\_\_bankssts & 0.471 & 225056.000 & 0.000 \\
	SurfArea\_\_bankssts & 0.471 & 268972.000 & 0.002 \\
	SurfArea\_\_insula & -0.463 & 16.000 & 0.000 \\
	Area\_\_insula & -0.463 & 106012.500 & 0.000 \\
	Thick\_\_\_lateraloccipital & -0.462 & 0.000 & 0.000 \\
	GrayVol\_\_parahippocampal & 0.458 & 198655.000 & 0.000 \\
	SurfArea\_\_isthmuscingulate & 0.446 & 0.000 & 0.000 \\
	Area\_\_isthmuscingulate & 0.446 & 184355.500 & 0.000 \\
	Thick\_\_\_transversetemporal & -0.443 & 0.000 & 0.000 \\
	SurfArea\_\_caudalmiddlefrontal & 0.433 & 66604.000 & 0.000 \\
	Area\_\_caudalmiddlefrontal & 0.433 & 179733.500 & 0.000 \\
	ThickAvg\_\_rostralmiddlefrontal & -0.431 & 0.000 & 0.000 \\
	GrayVol\_\_precuneus & -0.410 & 96547.500 & 0.000 \\
	Thick\_\_\_superiortemporal & -0.403 & 0.000 & 0.000 \\
	SurfArea\_\_superiortemporal & 0.400 & 5106.000 & 0.000 \\
	Area\_\_superiortemporal & 0.400 & 204111.000 & 0.000 \\
	SurfArea\_\_supramarginal & 0.398 & 42292.500 & 0.000 \\
	Area\_\_supramarginal & 0.398 & 170684.500 & 0.000 \\
	Thick\_\_\_supramarginal & -0.396 & 0.000 & 0.000 \\
	SurfArea\_\_precuneus & -0.393 & 0.000 & 0.000 \\
	Area\_\_precuneus & -0.393 & 85970.000 & 0.000 \\
	Thick\_\_\_superiorparietal & -0.376 & 106136.500 & 0.000 \\
	GrayVol\_\_temporalpole & 0.374 & 52650.000 & 0.000 \\
	SurfArea\_\_postcentral & 0.355 & 0.000 & 0.000 \\
	Area\_\_postcentral & 0.355 & 182161.500 & 0.000 \\
	GrayVol\_\_rostralmiddlefrontal & -0.349 & 144353.000 & 0.000 \\
	GrayVol\_\_insula & -0.343 & 95660.500 & 0.000 \\
	Thick\_\_\_parsopercularis & -0.342 & 0.000 & 0.000 \\
	Thick\_\_\_parsorbitalis & -0.339 & 0.000 & 0.000 \\
	GrayVol\_\_isthmuscingulate & 0.334 & 154757.000 & 0.000 \\
	GrayVol\_\_supramarginal & 0.333 & 144106.500 & 0.000 \\
	GrayVol\_\_caudalmiddlefrontal & 0.313 & 137658.000 & 0.000 \\
	Thick\_\_\_inferiortemporal & -0.305 & 122975.000 & 0.000 \\
	Thick\_\_\_insula & 0.302 & 0.000 & 0.000 \\
	Thick\_\_\_precuneus & -0.302 & 0.000 & 0.000 \\
	SurfArea\_\_inferiortemporal & 0.286 & 0.000 & 0.000 \\
	Area\_\_inferiortemporal & 0.286 & 245206.000 & 0.000 \\
	Area\_\_rostralmiddlefrontal & -0.286 & 115219.500 & 0.000 \\
	SurfArea\_\_rostralmiddlefrontal & -0.286 & 188723.500 & 0.000 \\
	SurfArea\_\_parahippocampal & 0.275 & 3113.500 & 0.000 \\
	Area\_\_parahippocampal & 0.275 & 140248.500 & 0.000 \\
	GrayVol\_\_bankssts & 0.266 & 152069.500 & 0.000 \\
	GrayVol\_\_postcentral & 0.265 & 140871.500 & 0.000 \\
	Thick\_\_\_superiorfrontal & -0.248 & 2378.500 & 0.000 \\
	Thick\_\_\_posteriorcingulate & 0.242 & 9243.000 & 0.000 \\
	Thick\_\_\_paracentral & -0.221 & 2957.500 & 0.000 \\
	Thick\_\_\_postcentral & -0.218 & 0.000 & 0.000 \\
	GrayVol\_\_cuneus & -0.217 & 208953.500 & 0.000 \\
	GrayVol\_\_fusiform & 0.216 & 212240.500 & 0.000 \\
	SurfArea\_\_lateralorbitofrontal & 0.211 & 110.000 & 0.000 \\
	Area\_\_lateralorbitofrontal & 0.211 & 194772.500 & 0.000 \\
	SurfArea\_\_fusiform & 0.206 & 0.000 & 0.000 \\
	Area\_\_fusiform & 0.206 & 208310.000 & 0.000 \\
	ThickAvg\_\_superiorfrontal & -0.195 & 0.000 & 0.000 \\
	GrayVol\_\_superiortemporal & 0.190 & 106677.000 & 0.000 \\
	SurfArea\_\_posteriorcingulate & -0.187 & 6120.500 & 0.000 \\
	Area\_\_posteriorcingulate & -0.187 & 275233.000 & 0.010 \\
	Thick\_\_\_isthmuscingulate & -0.183 & 0.000 & 0.000 \\
	SurfArea\_\_cuneus & -0.179 & 1519.000 & 0.000 \\
	Area\_\_cuneus & -0.179 & 191317.500 & 0.000 \\
	Area\_\_superiorfrontal & 0.175 & 250741.000 & 0.000 \\
	SurfArea\_\_superiorfrontal & 0.175 & 254495.000 & 0.000 \\
	GrayVol\_\_superiorparietal & -0.174 & 307024.500 & 0.784 \\
	SurfArea\_\_lateraloccipital & 0.164 & 66088.000 & 0.000 \\
	Area\_\_lateraloccipital & 0.164 & 297578.500 & 0.744 \\
	Thick\_\_\_fusiform & -0.158 & 0.000 & 0.000 \\
	GrayVol\_\_lateralorbitofrontal & 0.157 & 170688.000 & 0.000 \\
	Thick\_\_\_caudalmiddlefrontal & -0.153 & 0.000 & 0.000 \\
	Thick\_\_\_lateralorbitofrontal & -0.149 & 0.000 & 0.000 \\
	SurfArea\_\_precentral & -0.140 & 10.000 & 0.000 \\
	Area\_\_precentral & -0.140 & 262601.500 & 0.000 \\
	Thick\_\_\_lingual & -0.136 & 22381.500 & 0.000 \\
	GrayVol\_\_inferiortemporal & 0.135 & 177113.000 & 0.000 \\
	Thick\_\_\_rostralmiddlefrontal & -0.129 & 0.000 & 0.000 \\
	ThickAvg\_\_bankssts & 0.127 & 0.000 & 0.000 \\
	Thick\_\_\_frontalpole & -0.119 & 0.000 & 0.000 \\
	GrayVol\_\_posteriorcingulate & -0.099 & 246876.500 & 0.000 \\
	GrayVol\_\_superiorfrontal & 0.094 & 195883.000 & 0.000 \\
	GrayVol\_\_precentral & -0.088 & 232292.500 & 0.000 \\
	Thick\_\_\_parahippocampal & 0.087 & 0.000 & 0.000 \\
	GrayVol\_\_lingual & -0.063 & 290324.500 & 0.268 \\
	Thick\_\_\_rostralanteriorcingulate & 0.063 & 0.000 & 0.000 \\
	Thick\_\_\_cuneus & -0.063 & 132102.000 & 0.000 \\
	SurfArea\_\_lingual & -0.044 & 3167.500 & 0.000 \\
	Area\_\_lingual & -0.044 & 284216.000 & 0.112 \\
	GrayVol\_\_medialorbitofrontal & -0.044 & 128854.000 & 0.000 \\
	ThickAvg\_\_parsopercularis & 0.040 & 0.000 & 0.000 \\
	Thick\_\_\_pericalcarine & -0.031 & 4999.500 & 0.000 \\
	Thick\_\_\_precentral & 0.029 & 4019.500 & 0.000 \\
	GrayVol\_\_lateraloccipital & 0.024 & 232025.500 & 0.000 \\
	SurfArea\_\_superiorparietal & -0.009 & 592.500 & 0.000 \\
	Area\_\_superiorparietal & -0.009 & 215625.000 & 0.000 \\
	\bottomrule
\end{tabular}
\footnotesize
\caption{Measures of Separation of base FreeSurfer Features (left vs.\ right
hemisphere). d = Cohen's d, W = Wilcoxon signed rank test, W (p) = p-value
for W. Note: p-values were adjusted for multiple comparisons using the
Holm-Bonferroni stepdown method
}
\normalsize
\label{tab:lateral-fs}
\end{table}

\section{Appendix B} \label{sec:appendix-b}

% generated by src/munging/hcp.py::load_HCP_complete

In the tables below, names follow HCP data naming conventions, unless otherwise
indicated. The right column label indicates the name given to the single-factor
synthetic target, and the right column values are the factor loadings. Factor
loadings are the Pearson correlation between the original, unreduced variable
(e.g. "gambling\_task\_perc\_larger", below) and the final linear factor reduction
(e.g. "gambling\_perf", directly below).

\_perf = test performance, i.e.\ test score
\_rt = reaction time, i.e.\ reaction times on a test
wm = working memory

\begin{table}
\centering
\begin{tabular}{lr}
\toprule
 & gambling\_perf \\
\midrule
gambling\_task\_perc\_larger & -1.0000 \\
gambling\_task\_reward\_perc\_larger & -0.8926 \\
gambling\_task\_punish\_perc\_larger & -0.8337 \\
\bottomrule
\end{tabular}

\footnotesize
\caption{Synthetic target \texttt{gambling-perf} factor loadings.}
\normalsize
\label{tab:gambling-perf}
\end{table}


\begin{table}
\centering
\begin{tabular}{lr}
\toprule
 & emotion\_perf \\
\midrule
emotion\_task\_acc & -1.0000 \\
emotion\_task\_shape\_acc & -0.8952 \\
emotion\_task\_face\_acc & -0.8298 \\
\bottomrule
\end{tabular}

\footnotesize
\caption{Synthetic target \texttt{emotion-perf} factor loadings.}
\normalsize
\label{tab:emotion-perf}
\end{table}


\begin{table}
\centering
\begin{tabular}{lr}
\toprule
 & language\_rt \\
\midrule
language\_task\_median\_rt & 1.0000 \\
language\_task\_story\_median\_rt & 0.8261 \\
language\_task\_math\_median\_rt & 0.8138 \\
\bottomrule
\end{tabular}

\footnotesize
\caption{Synthetic target \texttt{language-rt} factor loadings.}
\normalsize
\label{tab:language-rt}
\end{table}


\begin{table}
\centering
\begin{tabular}{lr}
\toprule
 & relational\_rt \\
\midrule
relational\_task\_median\_rt & -1.0000 \\
relational\_task\_rel\_median\_rt & -0.9503 \\
relational\_task\_match\_median\_rt & -0.8610 \\
\bottomrule
\end{tabular}

\footnotesize
\caption{Synthetic target \texttt{relational-rt} factor loadings.}
\normalsize
\label{tab:relational-rt}
\end{table}


\begin{table}
\centering
\begin{tabular}{lr}
\toprule
 & emotion\_rt \\
\midrule
emotion\_task\_median\_rt & 1.0000 \\
emotion\_task\_face\_median\_rt & 0.9528 \\
emotion\_task\_shape\_median\_rt & 0.9373 \\
\bottomrule
\end{tabular}

\footnotesize
\caption{Synthetic target \texttt{emotion-rt} factor loadings.}
\normalsize
\label{tab:emotion-rt}
\end{table}


\begin{table}
\centering
\begin{tabular}{lr}
\toprule
 & language\_perf \\
\midrule
language\_task\_math\_acc & -0.9975 \\
language\_task\_acc & -0.8386 \\
language\_task\_story\_avg\_difficulty\_level & -0.7606 \\
\bottomrule
\end{tabular}

\footnotesize
\caption{Synthetic target \texttt{language-perf} factor loadings.}
\normalsize
\label{tab:language-perf}
\end{table}


\begin{table}
\centering
\begin{tabular}{lr}
\toprule
 & p\_matrices \\
\midrule
p\_matrices & 1.0000 \\
pmat24\_a\_cr & 0.7158 \\
pmat24\_a\_si & -0.7005 \\
\bottomrule
\end{tabular}

\footnotesize
\caption{Synthetic target \texttt{p-matrices} factor loadings.}
\normalsize
\label{tab:p-matrices}
\end{table}


\begin{table}
\centering
\begin{tabular}{lr}
\toprule
 & social\_rt \\
\midrule
social\_task\_median\_rt\_random & 0.9838 \\
social\_task\_random\_median\_rt\_random & 0.9826 \\
social\_task\_tom\_median\_rt\_tom & 0.4105 \\
social\_task\_median\_rt\_tom & 0.4017 \\
\bottomrule
\end{tabular}

\footnotesize
\caption{Synthetic target \texttt{social-rt} factor loadings.}
\normalsize
\label{tab:social-rt}
\end{table}


\begin{table}
\centering
\begin{tabular}{lr}
\toprule
 & psqi\_latent \\
\midrule
psqi\_score & 1.0000 \\
psqi\_comp1 & 0.6853 \\
psqi\_comp4 & 0.6379 \\
psqi\_comp3 & 0.6266 \\
psqi\_comp7 & 0.4649 \\
\bottomrule
\end{tabular}

\footnotesize
\caption{Synthetic target \texttt{psqi-latent} factor loadings.}
\normalsize
\label{tab:psqi-latent}
\end{table}


\begin{table}
\centering
\begin{tabular}{lr}
\toprule
 & gambling\_rt \\
\midrule
gambling\_task\_median\_rt\_smaller & 1.0000 \\
gambling\_task\_reward\_median\_rt\_smaller & 0.9512 \\
gambling\_task\_punish\_median\_rt\_smaller & 0.9504 \\
gambling\_task\_median\_rt\_larger & 0.8836 \\
gambling\_task\_reward\_median\_rt\_larger & 0.8451 \\
gambling\_task\_punish\_median\_rt\_larger & 0.8357 \\
\bottomrule
\end{tabular}

\footnotesize
\caption{Synthetic target \texttt{gambling-rt} factor loadings.}
\normalsize
\label{tab:gambling-rt}
\end{table}


\begin{table}
\centering
\begin{tabular}{lr}
\toprule
 & social\_random\_perf \\
\midrule
social\_task\_random\_perc\_random & -1.0000 \\
social\_task\_perc\_random & -0.9096 \\
social\_task\_random\_perc\_unsure & 0.8121 \\
social\_task\_perc\_unsure & 0.7238 \\
social\_task\_random\_perc\_tom & 0.5595 \\
social\_task\_perc\_tom & 0.3081 \\
social\_task\_tom\_perc\_tom & -0.1047 \\
social\_task\_tom\_perc\_unsure & 0.0870 \\
social\_task\_tom\_perc\_random & 0.0638 \\
\bottomrule
\end{tabular}

\footnotesize
\caption{Synthetic target \texttt{social-random-perf} factor loadings.}
\normalsize
\label{tab:social-random-perf}
\end{table}


\begin{table}
\centering
\begin{tabular}{lr}
\toprule
 & int\_g\_like \\
\midrule
cogtotalcomp\_unadj & -1.0000 \\
cogearlycomp\_unadj & -0.8750 \\
cogfluidcomp\_unadj & -0.8501 \\
cogcrystalcomp\_unadj & -0.7735 \\
readeng\_unadj & -0.7230 \\
picvocab\_unadj & -0.6767 \\
listsort\_unadj & -0.5698 \\
cardsort\_unadj & -0.5668 \\
procspeed\_unadj & -0.5640 \\
flanker\_unadj & -0.5116 \\
picseq\_unadj & -0.4938 \\
\bottomrule
\end{tabular}

\footnotesize
\caption{Synthetic target \texttt{int-g-like} factor loadings.}
\normalsize
\label{tab:int-g-like}
\end{table}


\begin{table}
\centering
\begin{tabular}{lr}
\toprule
 & neg\_emotionality \\
\midrule
percstress\_unadj & 0.8250 \\
sadness\_unadj & 0.8165 \\
loneliness\_unadj & 0.7812 \\
neofac\_n & 0.7596 \\
angaffect\_unadj & 0.7127 \\
fearaffect\_unadj & 0.7000 \\
percreject\_unadj & 0.6994 \\
lifesatisf\_unadj & -0.6571 \\
posaffect\_unadj & -0.6541 \\
anghostil\_unadj & 0.6524 \\
emotsupp\_unadj & -0.6347 \\
meanpurp\_unadj & -0.6078 \\
friendship\_unadj & -0.5947 \\
selfeff\_unadj & -0.5589 \\
perchostil\_unadj & 0.5512 \\
neofac\_e & -0.4649 \\
instrusupp\_unadj & -0.4578 \\
neofac\_a & -0.3779 \\
fearsomat\_unadj & 0.3752 \\
\bottomrule
\end{tabular}

\footnotesize
\caption{Synthetic target \texttt{neg-emotionality} factor loadings.}
\normalsize
\label{tab:neg-emotionality}
\end{table}


\begin{table}
\centering
\begin{tabular}{lr}
\toprule
 & wm\_rt \\
\midrule
wm\_task\_median\_rt & 1.0000 \\
wm\_task\_0bk\_median\_rt & 0.9059 \\
wm\_task\_2bk\_median\_rt & 0.8999 \\
wm\_task\_0bk\_face\_median\_rt & 0.7928 \\
wm\_task\_0bk\_tool\_median\_rt & 0.7924 \\
wm\_task\_0bk\_body\_median\_rt & 0.7914 \\
wm\_task\_0bk\_place\_median\_rt & 0.7913 \\
wm\_task\_0bk\_tool\_median\_rt\_nontarget & 0.7911 \\
wm\_task\_2bk\_face\_median\_rt & 0.7886 \\
wm\_task\_0bk\_face\_median\_rt\_nontarget & 0.7847 \\
wm\_task\_0bk\_place\_median\_rt\_nontarget & 0.7826 \\
wm\_task\_0bk\_body\_median\_rt\_nontarget & 0.7816 \\
wm\_task\_2bk\_place\_median\_rt & 0.7638 \\
wm\_task\_2bk\_face\_median\_rt\_nontarget & 0.7573 \\
wm\_task\_2bk\_tool\_median\_rt & 0.7409 \\
wm\_task\_2bk\_place\_median\_rt\_nontarget & 0.7372 \\
wm\_task\_2bk\_tool\_median\_rt\_nontarget & 0.7083 \\
wm\_task\_2bk\_body\_median\_rt & 0.6629 \\
wm\_task\_2bk\_tool\_median\_rt\_target & 0.6306 \\
wm\_task\_2bk\_body\_median\_rt\_nontarget & 0.6159 \\
wm\_task\_0bk\_face\_median\_rt\_target & 0.5825 \\
wm\_task\_2bk\_face\_median\_rt\_target & 0.5699 \\
wm\_task\_0bk\_place\_median\_rt\_target & 0.5302 \\
\bottomrule
\end{tabular}

\footnotesize
\caption{Synthetic target \texttt{wm-rt} factor loadings.}
\normalsize
\label{tab:wm-rt}
\end{table}


\begin{table}
\centering
\begin{tabular}{lr}
\toprule
 & wm\_perf \\
\midrule
wm\_task\_acc & -0.9931 \\
wm\_task\_0bk\_acc & -0.8816 \\
wm\_task\_2bk\_acc & -0.7975 \\
wm\_task\_0bk\_body\_acc & -0.7361 \\
wm\_task\_0bk\_tool\_acc & -0.7269 \\
wm\_task\_0bk\_place\_acc & -0.6881 \\
wm\_task\_0bk\_tool\_acc\_nontarget & -0.6827 \\
wm\_task\_0bk\_body\_acc\_nontarget & -0.6796 \\
wm\_task\_0bk\_face\_acc & -0.6730 \\
wm\_task\_2bk\_body\_acc & -0.6715 \\
wm\_task\_0bk\_face\_acc\_nontarget & -0.6543 \\
wm\_task\_0bk\_place\_acc\_nontarget & -0.6469 \\
wm\_task\_2bk\_face\_acc & -0.6418 \\
wm\_task\_2bk\_tool\_acc & -0.6395 \\
wm\_task\_2bk\_place\_acc & -0.6326 \\
wm\_task\_0bk\_body\_acc\_target & -0.6263 \\
wm\_task\_0bk\_tool\_acc\_target & -0.5900 \\
wm\_task\_2bk\_body\_acc\_nontarget & -0.5782 \\
wm\_task\_0bk\_face\_acc\_target & -0.5502 \\
wm\_task\_0bk\_place\_acc\_target & -0.5501 \\
wm\_task\_2bk\_body\_acc\_target & -0.5481 \\
wm\_task\_2bk\_tool\_acc\_nontarget & -0.5383 \\
wm\_task\_2bk\_tool\_acc\_target & -0.4792 \\
wm\_task\_2bk\_face\_acc\_target & -0.4539 \\
wm\_task\_2bk\_place\_acc\_target & -0.4516 \\
\bottomrule
\end{tabular}

\footnotesize
\caption{Synthetic target \texttt{wm-perf} factor loadings.}
\normalsize
\label{tab:wm-perf}
\end{table}

\section{Appendix C} \label{sec:appendix-c}

\begin{table}
\centering
\begin{tabular}{llllrrr}
	\toprule
	target & feats & selection & model & r2 & mae & mae+ \\
	\midrule
	int\_g\_like & FS & none & elastic & 0.144 & 0.184 & 0.018 \\
	int\_g\_like & FS & assoc & elastic & 0.143 & 0.184 & 0.018 \\
	int\_g\_like & FS & pred & elastic & 0.143 & 0.184 & 0.018 \\
	int\_g\_like & FS & embed\_linear & elastic & 0.141 & 0.184 & 0.018 \\
	int\_g\_like & FS & embed\_lgbm & elastic & 0.141 & 0.189 & 0.013 \\
	int\_g\_like & FS & embed\_linear & lgbm & 0.132 & 0.191 & 0.011 \\
	int\_g\_like & FS & pred & lgbm & 0.127 & 0.191 & 0.011 \\
	int\_g\_like & FS & none & lgbm & 0.116 & 0.191 & 0.011 \\
	int\_g\_like & FS+CMC & assoc & lgbm & 0.110 & 0.193 & 0.009 \\
	language\_perf & FS & embed\_linear & lgbm & 0.106 & 0.194 & 0.010 \\
	int\_g\_like & CMC & embed\_lgbm & elastic & 0.104 & 0.191 & 0.011 \\
	int\_g\_like & FS+CMC & embed\_lgbm & elastic & 0.104 & 0.192 & 0.010 \\
	language\_perf & FS & embed\_lgbm & elastic & 0.101 & 0.198 & 0.006 \\
	int\_g\_like & FS & embed\_lgbm & lgbm & 0.100 & 0.193 & 0.009 \\
	language\_perf & CMC & embed\_linear & lgbm & 0.098 & 0.197 & 0.007 \\
	wm\_perf & FS & embed\_lgbm & lgbm & 0.097 & 0.196 & 0.003 \\
	language\_perf & FS & none & lgbm & 0.095 & 0.195 & 0.008 \\
	wm\_perf & FS & embed\_lgbm & elastic & 0.094 & 0.197 & 0.002 \\
	language\_perf & CMC & assoc & lgbm & 0.093 & 0.196 & 0.007 \\
	wm\_perf & CMC & embed\_linear & lgbm & 0.093 & 0.197 & 0.002 \\
	wm\_perf & FS & pred & lgbm & 0.090 & 0.198 & 0.000 \\
	language\_perf & FS+CMC & embed\_linear & lgbm & 0.090 & 0.197 & 0.007 \\
	language\_perf & FS+CMC & assoc & lgbm & 0.089 & 0.197 & 0.007 \\
	language\_perf & FS+CMC & pred & lgbm & 0.087 & 0.197 & 0.007 \\
	int\_g\_like & CMC & pred & lgbm & 0.086 & 0.195 & 0.007 \\
	language\_perf & CMC & pred & lgbm & 0.086 & 0.197 & 0.007 \\
	language\_perf & CMC & none & lgbm & 0.085 & 0.197 & 0.006 \\
	language\_perf & CMC & none & knn & 0.084 & 0.198 & 0.006 \\
	int\_g\_like & FS+CMC & none & lgbm & 0.084 & 0.195 & 0.007 \\
	int\_g\_like & CMC & assoc & lgbm & 0.083 & 0.195 & 0.007 \\
	int\_g\_like & CMC & none & lgbm & 0.083 & 0.196 & 0.006 \\
	language\_perf & FS & pred & lgbm & 0.080 & 0.198 & 0.005 \\
	int\_g\_like & CMC & embed\_linear & lgbm & 0.080 & 0.196 & 0.006 \\
	wm\_perf & CMC & embed\_lgbm & elastic & 0.079 & 0.197 & 0.002 \\
	int\_g\_like & CMC & embed\_linear & elastic & 0.077 & 0.194 & 0.008 \\
	int\_g\_like & CMC & pred & elastic & 0.076 & 0.194 & 0.008 \\
	int\_g\_like & FS+CMC & pred & elastic & 0.076 & 0.191 & 0.011 \\
	int\_g\_like & FS+CMC & embed\_linear & knn & 0.073 & 0.195 & 0.007 \\
	int\_g\_like & FS & assoc & lgbm & 0.073 & 0.196 & 0.006 \\
	int\_g\_like & CMC & none & elastic & 0.073 & 0.194 & 0.008 \\
	int\_g\_like & CMC & assoc & elastic & 0.073 & 0.194 & 0.008 \\
	language\_perf & FS+CMC & embed\_lgbm & elastic & 0.073 & 0.201 & 0.003 \\
	int\_g\_like & FS+CMC & assoc & elastic & 0.072 & 0.192 & 0.010 \\
	int\_g\_like & FS+CMC & none & elastic & 0.071 & 0.192 & 0.010 \\
	int\_g\_like & FS+CMC & embed\_linear & lgbm & 0.071 & 0.196 & 0.006 \\
	language\_perf & FS & assoc & lgbm & 0.070 & 0.198 & 0.006 \\
	language\_perf & FS & wrap & elastic & 0.070 & 0.201 & 0.003 \\
	language\_perf & FS+CMC & wrap & elastic & 0.070 & 0.201 & 0.003 \\
	language\_perf & FS+CMC & none & lgbm & 0.068 & 0.200 & 0.003 \\
	int\_g\_like & FS+CMC & embed\_linear & elastic & 0.067 & 0.192 & 0.010 \\
	language\_perf & FS & pred & elastic & 0.065 & 0.203 & 0.001 \\
	language\_perf & CMC & embed\_lgbm & elastic & 0.057 & 0.202 & 0.002 \\
	language\_perf & FS+CMC & embed\_lgbm & lgbm & 0.056 & 0.201 & 0.002 \\
	language\_perf & CMC & embed\_lgbm & lgbm & 0.050 & 0.202 & 0.001 \\
	int\_g\_like & CMC & wrap & elastic & 0.050 & 0.198 & 0.004 \\
	int\_g\_like & FS+CMC & wrap & elastic & 0.050 & 0.198 & 0.004 \\
	int\_g\_like & FS & wrap & elastic & 0.050 & 0.198 & 0.004 \\
	language\_perf & FS & embed\_lgbm & lgbm & 0.049 & 0.200 & 0.003 \\
	language\_perf & CMC & wrap & elastic & 0.048 & 0.202 & 0.002 \\
	emotion\_rt & FS & embed\_lgbm & lgbm & 0.048 & 0.193 & 0.001 \\
	language\_perf & CMC & wrap & lgbm & 0.047 & 0.202 & 0.001 \\
	emotion\_rt & FS & none & lgbm & 0.046 & 0.190 & 0.004 \\
	int\_g\_like & FS & wrap & lgbm & 0.044 & 0.198 & 0.004 \\
	emotion\_rt & FS+CMC & none & lgbm & 0.044 & 0.190 & 0.003 \\
	int\_g\_like & FS+CMC & wrap & lgbm & 0.043 & 0.199 & 0.003 \\
	int\_g\_like & CMC & wrap & lgbm & 0.042 & 0.199 & 0.003 \\
	int\_g\_like & FS+CMC & pred & lgbm & 0.042 & 0.198 & 0.004 \\
	emotion\_rt & FS & embed\_linear & lgbm & 0.041 & 0.190 & 0.004 \\
	emotion\_rt & FS & assoc & lgbm & 0.037 & 0.191 & 0.003 \\
	int\_g\_like & CMC & embed\_lgbm & lgbm & 0.036 & 0.197 & 0.005 \\
	p\_matrices & FS+CMC & wrap & lgbm & 0.036 & 0.237 & 0.002 \\
	p\_matrices & FS+CMC & wrap & elastic & 0.035 & 0.235 & 0.003 \\
	p\_matrices & FS & wrap & elastic & 0.035 & 0.235 & 0.003 \\
	p\_matrices & CMC & wrap & elastic & 0.033 & 0.234 & 0.004 \\
	emotion\_rt & FS & wrap & elastic & 0.032 & 0.191 & 0.003 \\
	emotion\_rt & CMC & wrap & elastic & 0.032 & 0.191 & 0.003 \\
	emotion\_rt & FS+CMC & wrap & elastic & 0.032 & 0.191 & 0.003 \\
	emotion\_rt & FS+CMC & embed\_linear & lgbm & 0.032 & 0.191 & 0.003 \\
	int\_g\_like & FS+CMC & wrap & knn & 0.031 & 0.200 & 0.002 \\
	emotion\_rt & FS & pred & lgbm & 0.030 & 0.192 & 0.002 \\
	p\_matrices & FS & pred & lgbm & 0.030 & 0.235 & 0.003 \\
	emotion\_rt & CMC & embed\_linear & lgbm & 0.027 & 0.191 & 0.003 \\
	p\_matrices & FS & wrap & lgbm & 0.027 & 0.237 & 0.001 \\
	emotion\_rt & FS+CMC & embed\_lgbm & elastic & 0.026 & 0.193 & 0.001 \\
	emotion\_rt & FS+CMC & pred & lgbm & 0.026 & 0.192 & 0.002 \\
	wm\_rt & CMC & pred & lgbm & 0.025 & 0.195 & 0.002 \\
	emotion\_rt & CMC & assoc & lgbm & 0.025 & 0.192 & 0.002 \\
	p\_matrices & FS & embed\_linear & lgbm & 0.024 & 0.237 & 0.002 \\
	p\_matrices & FS & none & lgbm & 0.024 & 0.237 & 0.001 \\
	emotion\_rt & CMC & pred & lgbm & 0.023 & 0.192 & 0.002 \\
	p\_matrices & FS & assoc & lgbm & 0.023 & 0.237 & 0.001 \\
	p\_matrices & FS+CMC & assoc & knn & 0.021 & 0.237 & 0.001 \\
	gambling\_rt & FS+CMC & pred & lgbm & 0.021 & 0.200 & 0.006 \\
	wm\_rt & FS & none & lgbm & 0.021 & 0.195 & 0.002 \\
	gambling\_rt & FS+CMC & assoc & lgbm & 0.019 & 0.203 & 0.003 \\
	gambling\_rt & FS+CMC & embed\_linear & lgbm & 0.018 & 0.202 & 0.003 \\
	int\_g\_like & CMC & wrap & knn & 0.018 & 0.199 & 0.003 \\
	wm\_rt & FS+CMC & assoc & lgbm & 0.016 & 0.197 & 0.000 \\
	gambling\_rt & FS+CMC & wrap & elastic & 0.015 & 0.203 & 0.003 \\
	gambling\_rt & CMC & wrap & elastic & 0.015 & 0.203 & 0.003 \\
	p\_matrices & FS+CMC & none & lgbm & 0.015 & 0.238 & 0.001 \\
	p\_matrices & FS+CMC & assoc & lgbm & 0.015 & 0.237 & 0.002 \\
	gambling\_rt & FS & embed\_linear & lgbm & 0.014 & 0.203 & 0.003 \\
	wm\_rt & CMC & none & lgbm & 0.014 & 0.197 & 0.000 \\
	p\_matrices & CMC & embed\_lgbm & knn & 0.013 & 0.236 & 0.002 \\
	gambling\_rt & CMC & pred & lgbm & 0.013 & 0.203 & 0.002 \\
	p\_matrices & FS & embed\_lgbm & elastic & 0.013 & 0.238 & 0.000 \\
	gambling\_rt & FS+CMC & wrap & lgbm & 0.013 & 0.204 & 0.002 \\
	emotion\_rt & FS+CMC & embed\_linear & knn & 0.012 & 0.194 & 0.000 \\
	gambling\_rt & FS & wrap & elastic & 0.011 & 0.204 & 0.002 \\
	gambling\_rt & FS & wrap & lgbm & 0.011 & 0.204 & 0.002 \\
	emotion\_rt & CMC & wrap & lgbm & 0.010 & 0.194 & 0.000 \\
	emotion\_rt & FS & wrap & lgbm & 0.009 & 0.193 & 0.001 \\
	gambling\_rt & CMC & none & lgbm & 0.009 & 0.204 & 0.002 \\
	gambling\_rt & FS & none & lgbm & 0.008 & 0.205 & 0.000 \\
	gambling\_rt & CMC & embed\_linear & lgbm & 0.008 & 0.204 & 0.002 \\
	emotion\_rt & FS+CMC & wrap & lgbm & 0.007 & 0.194 & 0.000 \\
	gambling\_rt & CMC & assoc & lgbm & 0.006 & 0.205 & 0.001 \\
	gambling\_perf & FS & none & lgbm & 0.006 & 0.166 & 0.000 \\
	wm\_rt & CMC & wrap & elastic & 0.006 & 0.196 & 0.001 \\
	wm\_rt & FS+CMC & wrap & elastic & 0.006 & 0.196 & 0.001 \\
	gambling\_rt & CMC & embed\_lgbm & elastic & 0.005 & 0.205 & 0.001 \\
	emotion\_rt & FS+CMC & assoc & lgbm & 0.004 & 0.193 & 0.001 \\
	p\_matrices & CMC & assoc & lgbm & 0.004 & 0.237 & 0.001 \\
	gambling\_perf & CMC & wrap & elastic & 0.003 & 0.165 & 0.001 \\
	language\_rt & FS & wrap & lgbm & 0.002 & 0.190 & 0.000 \\
	social\_rt & FS & embed\_lgbm & lgbm & 0.001 & 0.194 & 0.000 \\
	social\_rt & FS & wrap & elastic & 0.001 & 0.194 & 0.000 \\
	gambling\_perf & FS & embed\_linear & lgbm & 0.001 & 0.166 & 0.000 \\
	\bottomrule
\end{tabular}
\footnotesize
\caption{Performance of models exceeding dummy model performance.
FS = FreeSurfer features; CMC = CMC features; FS+CMC = both FS and CMC features used;
wrap = forward stepwise feature selection with a linear model;
assoc = feature selection by univariate association (mutual information);
pred = feature selection by (linear) univariate prediction performance (accuracy);
none = no feature selection (all features used in model)
lgbm = LightGBM regressor; elastic = ElasticNet;
r2 = coefficient of determination; mae = mean absolute error;
mae+ = improvement in MAE relative to dummy model MAE.}
\normalsize
\label{tab:cmc-p-model-predictive}
\end{table}

% from summary.py

\begin{table}
\centering
\begin{tabular}{lllr}
\toprule
target & feats & model & exceeds\_dummy \\
\midrule
emotion\_rt & FS & lgbm & 1.000 \\
int\_g\_like & CMC & elastic & 1.000 \\
int\_g\_like & CMC & lgbm & 1.000 \\
int\_g\_like & FS & elastic & 1.000 \\
int\_g\_like & FS & lgbm & 1.000 \\
int\_g\_like & FS+CMC & elastic & 1.000 \\
language\_perf & CMC & lgbm & 1.000 \\
int\_g\_like & FS+CMC & lgbm & 0.833 \\
language\_perf & FS+CMC & lgbm & 0.833 \\
p\_matrices & FS & lgbm & 0.833 \\
language\_perf & FS & lgbm & 0.833 \\
emotion\_rt & FS+CMC & lgbm & 0.833 \\
gambling\_rt & CMC & lgbm & 0.667 \\
emotion\_rt & CMC & lgbm & 0.667 \\
gambling\_rt & FS+CMC & lgbm & 0.667 \\
gambling\_rt & FS & lgbm & 0.500 \\
language\_perf & FS & elastic & 0.500 \\
p\_matrices & FS+CMC & lgbm & 0.500 \\
language\_perf & FS+CMC & elastic & 0.333 \\
language\_perf & CMC & elastic & 0.333 \\
wm\_perf & FS & lgbm & 0.333 \\
gambling\_rt & CMC & elastic & 0.333 \\
gambling\_perf & FS & lgbm & 0.333 \\
emotion\_rt & FS+CMC & elastic & 0.333 \\
p\_matrices & FS & elastic & 0.333 \\
wm\_rt & CMC & lgbm & 0.333 \\
wm\_perf & CMC & lgbm & 0.167 \\
wm\_perf & CMC & elastic & 0.167 \\
social\_rt & FS & lgbm & 0.167 \\
social\_rt & FS & elastic & 0.167 \\
p\_matrices & CMC & elastic & 0.167 \\
language\_rt & FS & lgbm & 0.167 \\
p\_matrices & CMC & lgbm & 0.167 \\
p\_matrices & FS+CMC & elastic & 0.167 \\
wm\_perf & FS & elastic & 0.167 \\
wm\_rt & FS+CMC & lgbm & 0.167 \\
gambling\_perf & CMC & elastic & 0.167 \\
emotion\_rt & CMC & elastic & 0.167 \\
gambling\_rt & FS+CMC & elastic & 0.167 \\
gambling\_rt & FS & elastic & 0.167 \\
wm\_rt & CMC & elastic & 0.167 \\
wm\_rt & FS+CMC & elastic & 0.167 \\
wm\_rt & FS & lgbm & 0.167 \\
emotion\_rt & FS & elastic & 0.167 \\
relational\_rt & FS & lgbm & 0.000 \\
social\_random\_perf & CMC & elastic & 0.000 \\
social\_random\_perf & CMC & lgbm & 0.000 \\
wm\_rt & FS & elastic & 0.000 \\
wm\_perf & FS+CMC & elastic & 0.000 \\
relational\_rt & CMC & lgbm & 0.000 \\
social\_random\_perf & FS & elastic & 0.000 \\
social\_random\_perf & FS & lgbm & 0.000 \\
social\_rt & CMC & elastic & 0.000 \\
social\_rt & CMC & lgbm & 0.000 \\
wm\_perf & FS+CMC & lgbm & 0.000 \\
relational\_rt & FS & elastic & 0.000 \\
emotion\_perf & CMC & elastic & 0.000 \\
relational\_rt & CMC & elastic & 0.000 \\
language\_rt & FS+CMC & elastic & 0.000 \\
emotion\_perf & FS & elastic & 0.000 \\
emotion\_perf & FS & lgbm & 0.000 \\
emotion\_perf & FS+CMC & elastic & 0.000 \\
emotion\_perf & FS+CMC & lgbm & 0.000 \\
gambling\_perf & CMC & lgbm & 0.000 \\
gambling\_perf & FS & elastic & 0.000 \\
language\_rt & CMC & elastic & 0.000 \\
language\_rt & CMC & lgbm & 0.000 \\
language\_rt & FS & elastic & 0.000 \\
language\_rt & FS+CMC & lgbm & 0.000 \\
psqi\_latent & FS & lgbm & 0.000 \\
emotion\_perf & CMC & lgbm & 0.000 \\
neg\_emotionality & CMC & lgbm & 0.000 \\
neg\_emotionality & FS & elastic & 0.000 \\
neg\_emotionality & FS & lgbm & 0.000 \\
neg\_emotionality & FS+CMC & elastic & 0.000 \\
neg\_emotionality & FS+CMC & lgbm & 0.000 \\
psqi\_latent & CMC & elastic & 0.000 \\
psqi\_latent & CMC & lgbm & 0.000 \\
psqi\_latent & FS & elastic & 0.000 \\
neg\_emotionality & CMC & elastic & 0.000 \\
\bottomrule
\end{tabular}

\footnotesize
\caption{%
Proportion of model runs that exceed dummy performance.\
feats = feature set; model = predictive model;
exceeds\_dummy = proportion of runs with performance exceeding dummy models;
}
\normalsize
\label{tab:cmc-model-p-predictive}
\end{table}


\section{Examples of citations, figures, tables, references}
\label{sec:others}

\subsection{Citations}
Citations use \verb+natbib+. The documentation may be found at
\begin{center}
	\url{http://mirrors.ctan.org/macros/latex/contrib/natbib/natnotes.pdf}
\end{center}

Here is an example usage of the two main commands (\verb+citet+ and
\verb+citep+): Some people thought a thing
\citep{buyssePittsburghSleepQuality1989} but other people thought something
else \citep{cattellScientificUseFactor1978}. Many people have speculated that
if we knew exactly why \citet{cattellScientificUseFactor1978} thought
this\dots

\subsection{Figures}

See Figure~\ref{fig:fig1}. Here is how you add footnotes\footnote{Sample of the first footnote.}.

\begin{figure}
	\centering
	\fbox{\rule[-.5cm]{4cm}{4cm} \rule[-.5cm]{4cm}{0cm}}
	\caption{Sample figure caption.}
	\label{fig:fig1}
\end{figure}

% \subsection{Tables}
% See awesome Table~\ref{tab:table}.

% The documentation for \verb+booktabs+ (`Publication quality tables in LaTeX') is available from:
% \begin{center}
% 	\url{https://www.ctan.org/pkg/booktabs}
% \end{center}


% \begin{table}
% 	\caption{Sample table title}
% 	\centering
% 	\begin{tabular}{lll}
% 		\toprule
% 		\multicolumn{2}{c}{Part}                   \\
% 		\cmidrule(r){1-2}
% 		Name     & Description     & Size ($\mu$m) \\
% 		\midrule
% 		Dendrite & Input terminal  & $\sim$100     \\
% 		Axon     & Output terminal & $\sim$10      \\
% 		Soma     & Cell body       & up to $10^6$  \\
% 		\bottomrule
% 	\end{tabular}
% 	\label{tab:table}
% \end{table}



\bibliographystyle{unsrtnat}
\bibliography{references}  %%% Uncomment this line and comment out the ``thebibliography'' section below to use the external .bib file (using bibtex) .


%%% Uncomment this section and comment out the \bibliography{references} line above to use inline references.
% \begin{thebibliography}{1}

% 	\bibitem{kour2014real}
% 	George Kour and Raid Saabne.
% 	\newblock Real-time segmentation of on-line handwritten arabic script.
% 	\newblock In {\em Frontiers in Handwriting Recognition (ICFHR), 2014 14th
% 			International Conference on}, pages 417--422. IEEE, 2014.

% 	\bibitem{kour2014fast}
% 	George Kour and Raid Saabne.
% 	\newblock Fast classification of handwritten on-line arabic characters.
% 	\newblock In {\em Soft Computing and Pattern Recognition (SoCPaR), 2014 6th
% 			International Conference of}, pages 312--318. IEEE, 2014.

% 	\bibitem{keshet2016prediction}
% 	Keshet, Renato, Alina Maor, and George Kour.
% 	\newblock Prediction-Based, Prioritized Market-Share Insight Extraction.
% 	\newblock In {\em Advanced Data Mining and Applications (ADMA), 2016 12th International
%                       Conference of}, pages 81--94,2016.

% \end{thebibliography}


\end{document}
